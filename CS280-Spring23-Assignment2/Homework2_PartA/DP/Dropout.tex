\documentclass[11pt]{article}

    \usepackage[breakable]{tcolorbox}
    \usepackage{parskip} % Stop auto-indenting (to mimic markdown behaviour)
    

    % Basic figure setup, for now with no caption control since it's done
    % automatically by Pandoc (which extracts ![](path) syntax from Markdown).
    \usepackage{graphicx}
    % Maintain compatibility with old templates. Remove in nbconvert 6.0
    \let\Oldincludegraphics\includegraphics
    % Ensure that by default, figures have no caption (until we provide a
    % proper Figure object with a Caption API and a way to capture that
    % in the conversion process - todo).
    \usepackage{caption}
    \DeclareCaptionFormat{nocaption}{}
    \captionsetup{format=nocaption,aboveskip=0pt,belowskip=0pt}

    \usepackage{float}
    \floatplacement{figure}{H} % forces figures to be placed at the correct location
    \usepackage{xcolor} % Allow colors to be defined
    \usepackage{enumerate} % Needed for markdown enumerations to work
    \usepackage{geometry} % Used to adjust the document margins
    \usepackage{amsmath} % Equations
    \usepackage{amssymb} % Equations
    \usepackage{textcomp} % defines textquotesingle
    % Hack from http://tex.stackexchange.com/a/47451/13684:
    \AtBeginDocument{%
        \def\PYZsq{\textquotesingle}% Upright quotes in Pygmentized code
    }
    \usepackage{upquote} % Upright quotes for verbatim code
    \usepackage{eurosym} % defines \euro

    \usepackage{iftex}
    \ifPDFTeX
        \usepackage[T1]{fontenc}
        \IfFileExists{alphabeta.sty}{
              \usepackage{alphabeta}
          }{
              \usepackage[mathletters]{ucs}
              \usepackage[utf8x]{inputenc}
          }
    \else
        \usepackage{fontspec}
        \usepackage{unicode-math}
    \fi

    \usepackage{fancyvrb} % verbatim replacement that allows latex
    \usepackage{grffile} % extends the file name processing of package graphics
                         % to support a larger range
    \makeatletter % fix for old versions of grffile with XeLaTeX
    \@ifpackagelater{grffile}{2019/11/01}
    {
      % Do nothing on new versions
    }
    {
      \def\Gread@@xetex#1{%
        \IfFileExists{"\Gin@base".bb}%
        {\Gread@eps{\Gin@base.bb}}%
        {\Gread@@xetex@aux#1}%
      }
    }
    \makeatother
    \usepackage[Export]{adjustbox} % Used to constrain images to a maximum size
    \adjustboxset{max size={0.9\linewidth}{0.9\paperheight}}

    % The hyperref package gives us a pdf with properly built
    % internal navigation ('pdf bookmarks' for the table of contents,
    % internal cross-reference links, web links for URLs, etc.)
    \usepackage{hyperref}
    % The default LaTeX title has an obnoxious amount of whitespace. By default,
    % titling removes some of it. It also provides customization options.
    \usepackage{titling}
    \usepackage{longtable} % longtable support required by pandoc >1.10
    \usepackage{booktabs}  % table support for pandoc > 1.12.2
    \usepackage{array}     % table support for pandoc >= 2.11.3
    \usepackage{calc}      % table minipage width calculation for pandoc >= 2.11.1
    \usepackage[inline]{enumitem} % IRkernel/repr support (it uses the enumerate* environment)
    \usepackage[normalem]{ulem} % ulem is needed to support strikethroughs (\sout)
                                % normalem makes italics be italics, not underlines
    \usepackage{mathrsfs}
    

    
    % Colors for the hyperref package
    \definecolor{urlcolor}{rgb}{0,.145,.698}
    \definecolor{linkcolor}{rgb}{.71,0.21,0.01}
    \definecolor{citecolor}{rgb}{.12,.54,.11}

    % ANSI colors
    \definecolor{ansi-black}{HTML}{3E424D}
    \definecolor{ansi-black-intense}{HTML}{282C36}
    \definecolor{ansi-red}{HTML}{E75C58}
    \definecolor{ansi-red-intense}{HTML}{B22B31}
    \definecolor{ansi-green}{HTML}{00A250}
    \definecolor{ansi-green-intense}{HTML}{007427}
    \definecolor{ansi-yellow}{HTML}{DDB62B}
    \definecolor{ansi-yellow-intense}{HTML}{B27D12}
    \definecolor{ansi-blue}{HTML}{208FFB}
    \definecolor{ansi-blue-intense}{HTML}{0065CA}
    \definecolor{ansi-magenta}{HTML}{D160C4}
    \definecolor{ansi-magenta-intense}{HTML}{A03196}
    \definecolor{ansi-cyan}{HTML}{60C6C8}
    \definecolor{ansi-cyan-intense}{HTML}{258F8F}
    \definecolor{ansi-white}{HTML}{C5C1B4}
    \definecolor{ansi-white-intense}{HTML}{A1A6B2}
    \definecolor{ansi-default-inverse-fg}{HTML}{FFFFFF}
    \definecolor{ansi-default-inverse-bg}{HTML}{000000}

    % common color for the border for error outputs.
    \definecolor{outerrorbackground}{HTML}{FFDFDF}

    % commands and environments needed by pandoc snippets
    % extracted from the output of `pandoc -s`
    \providecommand{\tightlist}{%
      \setlength{\itemsep}{0pt}\setlength{\parskip}{0pt}}
    \DefineVerbatimEnvironment{Highlighting}{Verbatim}{commandchars=\\\{\}}
    % Add ',fontsize=\small' for more characters per line
    \newenvironment{Shaded}{}{}
    \newcommand{\KeywordTok}[1]{\textcolor[rgb]{0.00,0.44,0.13}{\textbf{{#1}}}}
    \newcommand{\DataTypeTok}[1]{\textcolor[rgb]{0.56,0.13,0.00}{{#1}}}
    \newcommand{\DecValTok}[1]{\textcolor[rgb]{0.25,0.63,0.44}{{#1}}}
    \newcommand{\BaseNTok}[1]{\textcolor[rgb]{0.25,0.63,0.44}{{#1}}}
    \newcommand{\FloatTok}[1]{\textcolor[rgb]{0.25,0.63,0.44}{{#1}}}
    \newcommand{\CharTok}[1]{\textcolor[rgb]{0.25,0.44,0.63}{{#1}}}
    \newcommand{\StringTok}[1]{\textcolor[rgb]{0.25,0.44,0.63}{{#1}}}
    \newcommand{\CommentTok}[1]{\textcolor[rgb]{0.38,0.63,0.69}{\textit{{#1}}}}
    \newcommand{\OtherTok}[1]{\textcolor[rgb]{0.00,0.44,0.13}{{#1}}}
    \newcommand{\AlertTok}[1]{\textcolor[rgb]{1.00,0.00,0.00}{\textbf{{#1}}}}
    \newcommand{\FunctionTok}[1]{\textcolor[rgb]{0.02,0.16,0.49}{{#1}}}
    \newcommand{\RegionMarkerTok}[1]{{#1}}
    \newcommand{\ErrorTok}[1]{\textcolor[rgb]{1.00,0.00,0.00}{\textbf{{#1}}}}
    \newcommand{\NormalTok}[1]{{#1}}

    % Additional commands for more recent versions of Pandoc
    \newcommand{\ConstantTok}[1]{\textcolor[rgb]{0.53,0.00,0.00}{{#1}}}
    \newcommand{\SpecialCharTok}[1]{\textcolor[rgb]{0.25,0.44,0.63}{{#1}}}
    \newcommand{\VerbatimStringTok}[1]{\textcolor[rgb]{0.25,0.44,0.63}{{#1}}}
    \newcommand{\SpecialStringTok}[1]{\textcolor[rgb]{0.73,0.40,0.53}{{#1}}}
    \newcommand{\ImportTok}[1]{{#1}}
    \newcommand{\DocumentationTok}[1]{\textcolor[rgb]{0.73,0.13,0.13}{\textit{{#1}}}}
    \newcommand{\AnnotationTok}[1]{\textcolor[rgb]{0.38,0.63,0.69}{\textbf{\textit{{#1}}}}}
    \newcommand{\CommentVarTok}[1]{\textcolor[rgb]{0.38,0.63,0.69}{\textbf{\textit{{#1}}}}}
    \newcommand{\VariableTok}[1]{\textcolor[rgb]{0.10,0.09,0.49}{{#1}}}
    \newcommand{\ControlFlowTok}[1]{\textcolor[rgb]{0.00,0.44,0.13}{\textbf{{#1}}}}
    \newcommand{\OperatorTok}[1]{\textcolor[rgb]{0.40,0.40,0.40}{{#1}}}
    \newcommand{\BuiltInTok}[1]{{#1}}
    \newcommand{\ExtensionTok}[1]{{#1}}
    \newcommand{\PreprocessorTok}[1]{\textcolor[rgb]{0.74,0.48,0.00}{{#1}}}
    \newcommand{\AttributeTok}[1]{\textcolor[rgb]{0.49,0.56,0.16}{{#1}}}
    \newcommand{\InformationTok}[1]{\textcolor[rgb]{0.38,0.63,0.69}{\textbf{\textit{{#1}}}}}
    \newcommand{\WarningTok}[1]{\textcolor[rgb]{0.38,0.63,0.69}{\textbf{\textit{{#1}}}}}


    % Define a nice break command that doesn't care if a line doesn't already
    % exist.
    \def\br{\hspace*{\fill} \\* }
    % Math Jax compatibility definitions
    \def\gt{>}
    \def\lt{<}
    \let\Oldtex\TeX
    \let\Oldlatex\LaTeX
    \renewcommand{\TeX}{\textrm{\Oldtex}}
    \renewcommand{\LaTeX}{\textrm{\Oldlatex}}
    % Document parameters
    % Document title
    \title{Dropout}
    
    
    
    
    
% Pygments definitions
\makeatletter
\def\PY@reset{\let\PY@it=\relax \let\PY@bf=\relax%
    \let\PY@ul=\relax \let\PY@tc=\relax%
    \let\PY@bc=\relax \let\PY@ff=\relax}
\def\PY@tok#1{\csname PY@tok@#1\endcsname}
\def\PY@toks#1+{\ifx\relax#1\empty\else%
    \PY@tok{#1}\expandafter\PY@toks\fi}
\def\PY@do#1{\PY@bc{\PY@tc{\PY@ul{%
    \PY@it{\PY@bf{\PY@ff{#1}}}}}}}
\def\PY#1#2{\PY@reset\PY@toks#1+\relax+\PY@do{#2}}

\@namedef{PY@tok@w}{\def\PY@tc##1{\textcolor[rgb]{0.73,0.73,0.73}{##1}}}
\@namedef{PY@tok@c}{\let\PY@it=\textit\def\PY@tc##1{\textcolor[rgb]{0.24,0.48,0.48}{##1}}}
\@namedef{PY@tok@cp}{\def\PY@tc##1{\textcolor[rgb]{0.61,0.40,0.00}{##1}}}
\@namedef{PY@tok@k}{\let\PY@bf=\textbf\def\PY@tc##1{\textcolor[rgb]{0.00,0.50,0.00}{##1}}}
\@namedef{PY@tok@kp}{\def\PY@tc##1{\textcolor[rgb]{0.00,0.50,0.00}{##1}}}
\@namedef{PY@tok@kt}{\def\PY@tc##1{\textcolor[rgb]{0.69,0.00,0.25}{##1}}}
\@namedef{PY@tok@o}{\def\PY@tc##1{\textcolor[rgb]{0.40,0.40,0.40}{##1}}}
\@namedef{PY@tok@ow}{\let\PY@bf=\textbf\def\PY@tc##1{\textcolor[rgb]{0.67,0.13,1.00}{##1}}}
\@namedef{PY@tok@nb}{\def\PY@tc##1{\textcolor[rgb]{0.00,0.50,0.00}{##1}}}
\@namedef{PY@tok@nf}{\def\PY@tc##1{\textcolor[rgb]{0.00,0.00,1.00}{##1}}}
\@namedef{PY@tok@nc}{\let\PY@bf=\textbf\def\PY@tc##1{\textcolor[rgb]{0.00,0.00,1.00}{##1}}}
\@namedef{PY@tok@nn}{\let\PY@bf=\textbf\def\PY@tc##1{\textcolor[rgb]{0.00,0.00,1.00}{##1}}}
\@namedef{PY@tok@ne}{\let\PY@bf=\textbf\def\PY@tc##1{\textcolor[rgb]{0.80,0.25,0.22}{##1}}}
\@namedef{PY@tok@nv}{\def\PY@tc##1{\textcolor[rgb]{0.10,0.09,0.49}{##1}}}
\@namedef{PY@tok@no}{\def\PY@tc##1{\textcolor[rgb]{0.53,0.00,0.00}{##1}}}
\@namedef{PY@tok@nl}{\def\PY@tc##1{\textcolor[rgb]{0.46,0.46,0.00}{##1}}}
\@namedef{PY@tok@ni}{\let\PY@bf=\textbf\def\PY@tc##1{\textcolor[rgb]{0.44,0.44,0.44}{##1}}}
\@namedef{PY@tok@na}{\def\PY@tc##1{\textcolor[rgb]{0.41,0.47,0.13}{##1}}}
\@namedef{PY@tok@nt}{\let\PY@bf=\textbf\def\PY@tc##1{\textcolor[rgb]{0.00,0.50,0.00}{##1}}}
\@namedef{PY@tok@nd}{\def\PY@tc##1{\textcolor[rgb]{0.67,0.13,1.00}{##1}}}
\@namedef{PY@tok@s}{\def\PY@tc##1{\textcolor[rgb]{0.73,0.13,0.13}{##1}}}
\@namedef{PY@tok@sd}{\let\PY@it=\textit\def\PY@tc##1{\textcolor[rgb]{0.73,0.13,0.13}{##1}}}
\@namedef{PY@tok@si}{\let\PY@bf=\textbf\def\PY@tc##1{\textcolor[rgb]{0.64,0.35,0.47}{##1}}}
\@namedef{PY@tok@se}{\let\PY@bf=\textbf\def\PY@tc##1{\textcolor[rgb]{0.67,0.36,0.12}{##1}}}
\@namedef{PY@tok@sr}{\def\PY@tc##1{\textcolor[rgb]{0.64,0.35,0.47}{##1}}}
\@namedef{PY@tok@ss}{\def\PY@tc##1{\textcolor[rgb]{0.10,0.09,0.49}{##1}}}
\@namedef{PY@tok@sx}{\def\PY@tc##1{\textcolor[rgb]{0.00,0.50,0.00}{##1}}}
\@namedef{PY@tok@m}{\def\PY@tc##1{\textcolor[rgb]{0.40,0.40,0.40}{##1}}}
\@namedef{PY@tok@gh}{\let\PY@bf=\textbf\def\PY@tc##1{\textcolor[rgb]{0.00,0.00,0.50}{##1}}}
\@namedef{PY@tok@gu}{\let\PY@bf=\textbf\def\PY@tc##1{\textcolor[rgb]{0.50,0.00,0.50}{##1}}}
\@namedef{PY@tok@gd}{\def\PY@tc##1{\textcolor[rgb]{0.63,0.00,0.00}{##1}}}
\@namedef{PY@tok@gi}{\def\PY@tc##1{\textcolor[rgb]{0.00,0.52,0.00}{##1}}}
\@namedef{PY@tok@gr}{\def\PY@tc##1{\textcolor[rgb]{0.89,0.00,0.00}{##1}}}
\@namedef{PY@tok@ge}{\let\PY@it=\textit}
\@namedef{PY@tok@gs}{\let\PY@bf=\textbf}
\@namedef{PY@tok@gp}{\let\PY@bf=\textbf\def\PY@tc##1{\textcolor[rgb]{0.00,0.00,0.50}{##1}}}
\@namedef{PY@tok@go}{\def\PY@tc##1{\textcolor[rgb]{0.44,0.44,0.44}{##1}}}
\@namedef{PY@tok@gt}{\def\PY@tc##1{\textcolor[rgb]{0.00,0.27,0.87}{##1}}}
\@namedef{PY@tok@err}{\def\PY@bc##1{{\setlength{\fboxsep}{\string -\fboxrule}\fcolorbox[rgb]{1.00,0.00,0.00}{1,1,1}{\strut ##1}}}}
\@namedef{PY@tok@kc}{\let\PY@bf=\textbf\def\PY@tc##1{\textcolor[rgb]{0.00,0.50,0.00}{##1}}}
\@namedef{PY@tok@kd}{\let\PY@bf=\textbf\def\PY@tc##1{\textcolor[rgb]{0.00,0.50,0.00}{##1}}}
\@namedef{PY@tok@kn}{\let\PY@bf=\textbf\def\PY@tc##1{\textcolor[rgb]{0.00,0.50,0.00}{##1}}}
\@namedef{PY@tok@kr}{\let\PY@bf=\textbf\def\PY@tc##1{\textcolor[rgb]{0.00,0.50,0.00}{##1}}}
\@namedef{PY@tok@bp}{\def\PY@tc##1{\textcolor[rgb]{0.00,0.50,0.00}{##1}}}
\@namedef{PY@tok@fm}{\def\PY@tc##1{\textcolor[rgb]{0.00,0.00,1.00}{##1}}}
\@namedef{PY@tok@vc}{\def\PY@tc##1{\textcolor[rgb]{0.10,0.09,0.49}{##1}}}
\@namedef{PY@tok@vg}{\def\PY@tc##1{\textcolor[rgb]{0.10,0.09,0.49}{##1}}}
\@namedef{PY@tok@vi}{\def\PY@tc##1{\textcolor[rgb]{0.10,0.09,0.49}{##1}}}
\@namedef{PY@tok@vm}{\def\PY@tc##1{\textcolor[rgb]{0.10,0.09,0.49}{##1}}}
\@namedef{PY@tok@sa}{\def\PY@tc##1{\textcolor[rgb]{0.73,0.13,0.13}{##1}}}
\@namedef{PY@tok@sb}{\def\PY@tc##1{\textcolor[rgb]{0.73,0.13,0.13}{##1}}}
\@namedef{PY@tok@sc}{\def\PY@tc##1{\textcolor[rgb]{0.73,0.13,0.13}{##1}}}
\@namedef{PY@tok@dl}{\def\PY@tc##1{\textcolor[rgb]{0.73,0.13,0.13}{##1}}}
\@namedef{PY@tok@s2}{\def\PY@tc##1{\textcolor[rgb]{0.73,0.13,0.13}{##1}}}
\@namedef{PY@tok@sh}{\def\PY@tc##1{\textcolor[rgb]{0.73,0.13,0.13}{##1}}}
\@namedef{PY@tok@s1}{\def\PY@tc##1{\textcolor[rgb]{0.73,0.13,0.13}{##1}}}
\@namedef{PY@tok@mb}{\def\PY@tc##1{\textcolor[rgb]{0.40,0.40,0.40}{##1}}}
\@namedef{PY@tok@mf}{\def\PY@tc##1{\textcolor[rgb]{0.40,0.40,0.40}{##1}}}
\@namedef{PY@tok@mh}{\def\PY@tc##1{\textcolor[rgb]{0.40,0.40,0.40}{##1}}}
\@namedef{PY@tok@mi}{\def\PY@tc##1{\textcolor[rgb]{0.40,0.40,0.40}{##1}}}
\@namedef{PY@tok@il}{\def\PY@tc##1{\textcolor[rgb]{0.40,0.40,0.40}{##1}}}
\@namedef{PY@tok@mo}{\def\PY@tc##1{\textcolor[rgb]{0.40,0.40,0.40}{##1}}}
\@namedef{PY@tok@ch}{\let\PY@it=\textit\def\PY@tc##1{\textcolor[rgb]{0.24,0.48,0.48}{##1}}}
\@namedef{PY@tok@cm}{\let\PY@it=\textit\def\PY@tc##1{\textcolor[rgb]{0.24,0.48,0.48}{##1}}}
\@namedef{PY@tok@cpf}{\let\PY@it=\textit\def\PY@tc##1{\textcolor[rgb]{0.24,0.48,0.48}{##1}}}
\@namedef{PY@tok@c1}{\let\PY@it=\textit\def\PY@tc##1{\textcolor[rgb]{0.24,0.48,0.48}{##1}}}
\@namedef{PY@tok@cs}{\let\PY@it=\textit\def\PY@tc##1{\textcolor[rgb]{0.24,0.48,0.48}{##1}}}

\def\PYZbs{\char`\\}
\def\PYZus{\char`\_}
\def\PYZob{\char`\{}
\def\PYZcb{\char`\}}
\def\PYZca{\char`\^}
\def\PYZam{\char`\&}
\def\PYZlt{\char`\<}
\def\PYZgt{\char`\>}
\def\PYZsh{\char`\#}
\def\PYZpc{\char`\%}
\def\PYZdl{\char`\$}
\def\PYZhy{\char`\-}
\def\PYZsq{\char`\'}
\def\PYZdq{\char`\"}
\def\PYZti{\char`\~}
% for compatibility with earlier versions
\def\PYZat{@}
\def\PYZlb{[}
\def\PYZrb{]}
\makeatother


    % For linebreaks inside Verbatim environment from package fancyvrb.
    \makeatletter
        \newbox\Wrappedcontinuationbox
        \newbox\Wrappedvisiblespacebox
        \newcommand*\Wrappedvisiblespace {\textcolor{red}{\textvisiblespace}}
        \newcommand*\Wrappedcontinuationsymbol {\textcolor{red}{\llap{\tiny$\m@th\hookrightarrow$}}}
        \newcommand*\Wrappedcontinuationindent {3ex }
        \newcommand*\Wrappedafterbreak {\kern\Wrappedcontinuationindent\copy\Wrappedcontinuationbox}
        % Take advantage of the already applied Pygments mark-up to insert
        % potential linebreaks for TeX processing.
        %        {, <, #, %, $, ' and ": go to next line.
        %        _, }, ^, &, >, - and ~: stay at end of broken line.
        % Use of \textquotesingle for straight quote.
        \newcommand*\Wrappedbreaksatspecials {%
            \def\PYGZus{\discretionary{\char`\_}{\Wrappedafterbreak}{\char`\_}}%
            \def\PYGZob{\discretionary{}{\Wrappedafterbreak\char`\{}{\char`\{}}%
            \def\PYGZcb{\discretionary{\char`\}}{\Wrappedafterbreak}{\char`\}}}%
            \def\PYGZca{\discretionary{\char`\^}{\Wrappedafterbreak}{\char`\^}}%
            \def\PYGZam{\discretionary{\char`\&}{\Wrappedafterbreak}{\char`\&}}%
            \def\PYGZlt{\discretionary{}{\Wrappedafterbreak\char`\<}{\char`\<}}%
            \def\PYGZgt{\discretionary{\char`\>}{\Wrappedafterbreak}{\char`\>}}%
            \def\PYGZsh{\discretionary{}{\Wrappedafterbreak\char`\#}{\char`\#}}%
            \def\PYGZpc{\discretionary{}{\Wrappedafterbreak\char`\%}{\char`\%}}%
            \def\PYGZdl{\discretionary{}{\Wrappedafterbreak\char`\$}{\char`\$}}%
            \def\PYGZhy{\discretionary{\char`\-}{\Wrappedafterbreak}{\char`\-}}%
            \def\PYGZsq{\discretionary{}{\Wrappedafterbreak\textquotesingle}{\textquotesingle}}%
            \def\PYGZdq{\discretionary{}{\Wrappedafterbreak\char`\"}{\char`\"}}%
            \def\PYGZti{\discretionary{\char`\~}{\Wrappedafterbreak}{\char`\~}}%
        }
        % Some characters . , ; ? ! / are not pygmentized.
        % This macro makes them "active" and they will insert potential linebreaks
        \newcommand*\Wrappedbreaksatpunct {%
            \lccode`\~`\.\lowercase{\def~}{\discretionary{\hbox{\char`\.}}{\Wrappedafterbreak}{\hbox{\char`\.}}}%
            \lccode`\~`\,\lowercase{\def~}{\discretionary{\hbox{\char`\,}}{\Wrappedafterbreak}{\hbox{\char`\,}}}%
            \lccode`\~`\;\lowercase{\def~}{\discretionary{\hbox{\char`\;}}{\Wrappedafterbreak}{\hbox{\char`\;}}}%
            \lccode`\~`\:\lowercase{\def~}{\discretionary{\hbox{\char`\:}}{\Wrappedafterbreak}{\hbox{\char`\:}}}%
            \lccode`\~`\?\lowercase{\def~}{\discretionary{\hbox{\char`\?}}{\Wrappedafterbreak}{\hbox{\char`\?}}}%
            \lccode`\~`\!\lowercase{\def~}{\discretionary{\hbox{\char`\!}}{\Wrappedafterbreak}{\hbox{\char`\!}}}%
            \lccode`\~`\/\lowercase{\def~}{\discretionary{\hbox{\char`\/}}{\Wrappedafterbreak}{\hbox{\char`\/}}}%
            \catcode`\.\active
            \catcode`\,\active
            \catcode`\;\active
            \catcode`\:\active
            \catcode`\?\active
            \catcode`\!\active
            \catcode`\/\active
            \lccode`\~`\~
        }
    \makeatother

    \let\OriginalVerbatim=\Verbatim
    \makeatletter
    \renewcommand{\Verbatim}[1][1]{%
        %\parskip\z@skip
        \sbox\Wrappedcontinuationbox {\Wrappedcontinuationsymbol}%
        \sbox\Wrappedvisiblespacebox {\FV@SetupFont\Wrappedvisiblespace}%
        \def\FancyVerbFormatLine ##1{\hsize\linewidth
            \vtop{\raggedright\hyphenpenalty\z@\exhyphenpenalty\z@
                \doublehyphendemerits\z@\finalhyphendemerits\z@
                \strut ##1\strut}%
        }%
        % If the linebreak is at a space, the latter will be displayed as visible
        % space at end of first line, and a continuation symbol starts next line.
        % Stretch/shrink are however usually zero for typewriter font.
        \def\FV@Space {%
            \nobreak\hskip\z@ plus\fontdimen3\font minus\fontdimen4\font
            \discretionary{\copy\Wrappedvisiblespacebox}{\Wrappedafterbreak}
            {\kern\fontdimen2\font}%
        }%

        % Allow breaks at special characters using \PYG... macros.
        \Wrappedbreaksatspecials
        % Breaks at punctuation characters . , ; ? ! and / need catcode=\active
        \OriginalVerbatim[#1,codes*=\Wrappedbreaksatpunct]%
    }
    \makeatother

    % Exact colors from NB
    \definecolor{incolor}{HTML}{303F9F}
    \definecolor{outcolor}{HTML}{D84315}
    \definecolor{cellborder}{HTML}{CFCFCF}
    \definecolor{cellbackground}{HTML}{F7F7F7}

    % prompt
    \makeatletter
    \newcommand{\boxspacing}{\kern\kvtcb@left@rule\kern\kvtcb@boxsep}
    \makeatother
    \newcommand{\prompt}[4]{
        {\ttfamily\llap{{\color{#2}[#3]:\hspace{3pt}#4}}\vspace{-\baselineskip}}
    }
    

    
    % Prevent overflowing lines due to hard-to-break entities
    \sloppy
    % Setup hyperref package
    \hypersetup{
      breaklinks=true,  % so long urls are correctly broken across lines
      colorlinks=true,
      urlcolor=urlcolor,
      linkcolor=linkcolor,
      citecolor=citecolor,
      }
    % Slightly bigger margins than the latex defaults
    
    \geometry{verbose,tmargin=1in,bmargin=1in,lmargin=1in,rmargin=1in}
    
    

\begin{document}
    
    \maketitle
    
    

    
    \hypertarget{dropout}{%
\section{Dropout}\label{dropout}}

Dropout {[}1{]} is a technique for regularizing neural networks by
randomly setting some output activations to zero during the forward
pass. In this exercise you will implement a dropout layer and modify
your fully-connected network to optionally use dropout.

{[}1{]} \href{https://arxiv.org/abs/1207.0580}{Geoffrey E. Hinton et al,
``Improving neural networks by preventing co-adaptation of feature
detectors'', arXiv 2012}

    \begin{tcolorbox}[breakable, size=fbox, boxrule=1pt, pad at break*=1mm,colback=cellbackground, colframe=cellborder]
\prompt{In}{incolor}{1}{\boxspacing}
\begin{Verbatim}[commandchars=\\\{\}]
\PY{c+c1}{\PYZsh{} As usual, a bit of setup}
\PY{k+kn}{from} \PY{n+nn}{\PYZus{}\PYZus{}future\PYZus{}\PYZus{}} \PY{k+kn}{import} \PY{n}{print\PYZus{}function}
\PY{k+kn}{import} \PY{n+nn}{time}
\PY{k+kn}{import} \PY{n+nn}{numpy} \PY{k}{as} \PY{n+nn}{np}
\PY{k+kn}{import} \PY{n+nn}{matplotlib}\PY{n+nn}{.}\PY{n+nn}{pyplot} \PY{k}{as} \PY{n+nn}{plt}
\PY{k+kn}{from} \PY{n+nn}{cs231n}\PY{n+nn}{.}\PY{n+nn}{classifiers}\PY{n+nn}{.}\PY{n+nn}{fc\PYZus{}net} \PY{k+kn}{import} \PY{o}{*}
\PY{k+kn}{from} \PY{n+nn}{cs231n}\PY{n+nn}{.}\PY{n+nn}{data\PYZus{}utils} \PY{k+kn}{import} \PY{n}{get\PYZus{}CIFAR10\PYZus{}data}
\PY{k+kn}{from} \PY{n+nn}{cs231n}\PY{n+nn}{.}\PY{n+nn}{gradient\PYZus{}check} \PY{k+kn}{import} \PY{n}{eval\PYZus{}numerical\PYZus{}gradient}\PY{p}{,} \PY{n}{eval\PYZus{}numerical\PYZus{}gradient\PYZus{}array}
\PY{k+kn}{from} \PY{n+nn}{cs231n}\PY{n+nn}{.}\PY{n+nn}{solver} \PY{k+kn}{import} \PY{n}{Solver}

\PY{o}{\PYZpc{}}\PY{k}{matplotlib} inline
\PY{n}{plt}\PY{o}{.}\PY{n}{rcParams}\PY{p}{[}\PY{l+s+s1}{\PYZsq{}}\PY{l+s+s1}{figure.figsize}\PY{l+s+s1}{\PYZsq{}}\PY{p}{]} \PY{o}{=} \PY{p}{(}\PY{l+m+mf}{10.0}\PY{p}{,} \PY{l+m+mf}{8.0}\PY{p}{)} \PY{c+c1}{\PYZsh{} set default size of plots}
\PY{n}{plt}\PY{o}{.}\PY{n}{rcParams}\PY{p}{[}\PY{l+s+s1}{\PYZsq{}}\PY{l+s+s1}{image.interpolation}\PY{l+s+s1}{\PYZsq{}}\PY{p}{]} \PY{o}{=} \PY{l+s+s1}{\PYZsq{}}\PY{l+s+s1}{nearest}\PY{l+s+s1}{\PYZsq{}}
\PY{n}{plt}\PY{o}{.}\PY{n}{rcParams}\PY{p}{[}\PY{l+s+s1}{\PYZsq{}}\PY{l+s+s1}{image.cmap}\PY{l+s+s1}{\PYZsq{}}\PY{p}{]} \PY{o}{=} \PY{l+s+s1}{\PYZsq{}}\PY{l+s+s1}{gray}\PY{l+s+s1}{\PYZsq{}}

\PY{c+c1}{\PYZsh{} for auto\PYZhy{}reloading external modules}
\PY{c+c1}{\PYZsh{} see http://stackoverflow.com/questions/1907993/autoreload\PYZhy{}of\PYZhy{}modules\PYZhy{}in\PYZhy{}ipython}
\PY{o}{\PYZpc{}}\PY{k}{load\PYZus{}ext} autoreload
\PY{o}{\PYZpc{}}\PY{k}{autoreload} 2

\PY{k}{def} \PY{n+nf}{rel\PYZus{}error}\PY{p}{(}\PY{n}{x}\PY{p}{,} \PY{n}{y}\PY{p}{)}\PY{p}{:}
\PY{+w}{  }\PY{l+s+sd}{\PYZdq{}\PYZdq{}\PYZdq{} returns relative error \PYZdq{}\PYZdq{}\PYZdq{}}
  \PY{k}{return} \PY{n}{np}\PY{o}{.}\PY{n}{max}\PY{p}{(}\PY{n}{np}\PY{o}{.}\PY{n}{abs}\PY{p}{(}\PY{n}{x} \PY{o}{\PYZhy{}} \PY{n}{y}\PY{p}{)} \PY{o}{/} \PY{p}{(}\PY{n}{np}\PY{o}{.}\PY{n}{maximum}\PY{p}{(}\PY{l+m+mf}{1e\PYZhy{}8}\PY{p}{,} \PY{n}{np}\PY{o}{.}\PY{n}{abs}\PY{p}{(}\PY{n}{x}\PY{p}{)} \PY{o}{+} \PY{n}{np}\PY{o}{.}\PY{n}{abs}\PY{p}{(}\PY{n}{y}\PY{p}{)}\PY{p}{)}\PY{p}{)}\PY{p}{)}
\end{Verbatim}
\end{tcolorbox}

    \begin{Verbatim}[commandchars=\\\{\}]
=========== You can safely ignore the message below if you are NOT working on
ConvolutionalNetworks.ipynb ===========
        You will need to compile a Cython extension for a portion of this
assignment.
        The instructions to do this will be given in a section of the notebook
below.
        There will be an option for Colab users and another for Jupyter (local)
users.
    \end{Verbatim}

    \begin{tcolorbox}[breakable, size=fbox, boxrule=1pt, pad at break*=1mm,colback=cellbackground, colframe=cellborder]
\prompt{In}{incolor}{2}{\boxspacing}
\begin{Verbatim}[commandchars=\\\{\}]
\PY{c+c1}{\PYZsh{} Load the (preprocessed) CIFAR10 data.}

\PY{n}{data} \PY{o}{=} \PY{n}{get\PYZus{}CIFAR10\PYZus{}data}\PY{p}{(}\PY{p}{)}
\PY{k}{for} \PY{n}{k}\PY{p}{,} \PY{n}{v} \PY{o+ow}{in} \PY{n}{data}\PY{o}{.}\PY{n}{items}\PY{p}{(}\PY{p}{)}\PY{p}{:}
  \PY{n+nb}{print}\PY{p}{(}\PY{l+s+s1}{\PYZsq{}}\PY{l+s+si}{\PYZpc{}s}\PY{l+s+s1}{: }\PY{l+s+s1}{\PYZsq{}} \PY{o}{\PYZpc{}} \PY{n}{k}\PY{p}{,} \PY{n}{v}\PY{o}{.}\PY{n}{shape}\PY{p}{)}
\end{Verbatim}
\end{tcolorbox}

    \begin{Verbatim}[commandchars=\\\{\}]
X\_train:  (49000, 3, 32, 32)
y\_train:  (49000,)
X\_val:  (1000, 3, 32, 32)
y\_val:  (1000,)
X\_test:  (1000, 3, 32, 32)
y\_test:  (1000,)
    \end{Verbatim}

    \hypertarget{dropout-forward-pass}{%
\section{Dropout forward pass}\label{dropout-forward-pass}}

In the file \texttt{cs231n/layers.py}, implement the forward pass for
dropout. Since dropout behaves differently during training and testing,
make sure to implement the operation for both modes.

Once you have done so, run the cell below to test your implementation.

    \begin{tcolorbox}[breakable, size=fbox, boxrule=1pt, pad at break*=1mm,colback=cellbackground, colframe=cellborder]
\prompt{In}{incolor}{6}{\boxspacing}
\begin{Verbatim}[commandchars=\\\{\}]
\PY{n}{np}\PY{o}{.}\PY{n}{random}\PY{o}{.}\PY{n}{seed}\PY{p}{(}\PY{l+m+mi}{231}\PY{p}{)}
\PY{n}{x} \PY{o}{=} \PY{n}{np}\PY{o}{.}\PY{n}{random}\PY{o}{.}\PY{n}{randn}\PY{p}{(}\PY{l+m+mi}{500}\PY{p}{,} \PY{l+m+mi}{500}\PY{p}{)} \PY{o}{+} \PY{l+m+mi}{10}

\PY{k}{for} \PY{n}{p} \PY{o+ow}{in} \PY{p}{[}\PY{l+m+mf}{0.25}\PY{p}{,} \PY{l+m+mf}{0.4}\PY{p}{,} \PY{l+m+mf}{0.7}\PY{p}{]}\PY{p}{:}
  \PY{n}{out}\PY{p}{,} \PY{n}{\PYZus{}} \PY{o}{=} \PY{n}{dropout\PYZus{}forward}\PY{p}{(}\PY{n}{x}\PY{p}{,} \PY{p}{\PYZob{}}\PY{l+s+s1}{\PYZsq{}}\PY{l+s+s1}{mode}\PY{l+s+s1}{\PYZsq{}}\PY{p}{:} \PY{l+s+s1}{\PYZsq{}}\PY{l+s+s1}{train}\PY{l+s+s1}{\PYZsq{}}\PY{p}{,} \PY{l+s+s1}{\PYZsq{}}\PY{l+s+s1}{p}\PY{l+s+s1}{\PYZsq{}}\PY{p}{:} \PY{n}{p}\PY{p}{\PYZcb{}}\PY{p}{)}
  \PY{n}{out\PYZus{}test}\PY{p}{,} \PY{n}{\PYZus{}} \PY{o}{=} \PY{n}{dropout\PYZus{}forward}\PY{p}{(}\PY{n}{x}\PY{p}{,} \PY{p}{\PYZob{}}\PY{l+s+s1}{\PYZsq{}}\PY{l+s+s1}{mode}\PY{l+s+s1}{\PYZsq{}}\PY{p}{:} \PY{l+s+s1}{\PYZsq{}}\PY{l+s+s1}{test}\PY{l+s+s1}{\PYZsq{}}\PY{p}{,} \PY{l+s+s1}{\PYZsq{}}\PY{l+s+s1}{p}\PY{l+s+s1}{\PYZsq{}}\PY{p}{:} \PY{n}{p}\PY{p}{\PYZcb{}}\PY{p}{)}

  \PY{n+nb}{print}\PY{p}{(}\PY{l+s+s1}{\PYZsq{}}\PY{l+s+s1}{Running tests with p = }\PY{l+s+s1}{\PYZsq{}}\PY{p}{,} \PY{n}{p}\PY{p}{)}
  \PY{n+nb}{print}\PY{p}{(}\PY{l+s+s1}{\PYZsq{}}\PY{l+s+s1}{Mean of input: }\PY{l+s+s1}{\PYZsq{}}\PY{p}{,} \PY{n}{x}\PY{o}{.}\PY{n}{mean}\PY{p}{(}\PY{p}{)}\PY{p}{)}
  \PY{n+nb}{print}\PY{p}{(}\PY{l+s+s1}{\PYZsq{}}\PY{l+s+s1}{Mean of train\PYZhy{}time output: }\PY{l+s+s1}{\PYZsq{}}\PY{p}{,} \PY{n}{out}\PY{o}{.}\PY{n}{mean}\PY{p}{(}\PY{p}{)}\PY{p}{)}
  \PY{n+nb}{print}\PY{p}{(}\PY{l+s+s1}{\PYZsq{}}\PY{l+s+s1}{Mean of test\PYZhy{}time output: }\PY{l+s+s1}{\PYZsq{}}\PY{p}{,} \PY{n}{out\PYZus{}test}\PY{o}{.}\PY{n}{mean}\PY{p}{(}\PY{p}{)}\PY{p}{)}
  \PY{n+nb}{print}\PY{p}{(}\PY{l+s+s1}{\PYZsq{}}\PY{l+s+s1}{Fraction of train\PYZhy{}time output set to zero: }\PY{l+s+s1}{\PYZsq{}}\PY{p}{,} \PY{p}{(}\PY{n}{out} \PY{o}{==} \PY{l+m+mi}{0}\PY{p}{)}\PY{o}{.}\PY{n}{mean}\PY{p}{(}\PY{p}{)}\PY{p}{)}
  \PY{n+nb}{print}\PY{p}{(}\PY{l+s+s1}{\PYZsq{}}\PY{l+s+s1}{Fraction of test\PYZhy{}time output set to zero: }\PY{l+s+s1}{\PYZsq{}}\PY{p}{,} \PY{p}{(}\PY{n}{out\PYZus{}test} \PY{o}{==} \PY{l+m+mi}{0}\PY{p}{)}\PY{o}{.}\PY{n}{mean}\PY{p}{(}\PY{p}{)}\PY{p}{)}
  \PY{n+nb}{print}\PY{p}{(}\PY{p}{)}
\end{Verbatim}
\end{tcolorbox}

    \begin{Verbatim}[commandchars=\\\{\}]
Running tests with p =  0.25
Mean of input:  10.000207878477502
Mean of train-time output:  10.014059116977283
Mean of test-time output:  10.000207878477502
Fraction of train-time output set to zero:  0.749784
Fraction of test-time output set to zero:  0.0

Running tests with p =  0.4
Mean of input:  10.000207878477502
Mean of train-time output:  9.977917658761159
Mean of test-time output:  10.000207878477502
Fraction of train-time output set to zero:  0.600796
Fraction of test-time output set to zero:  0.0

Running tests with p =  0.7
Mean of input:  10.000207878477502
Mean of train-time output:  9.987811912159426
Mean of test-time output:  10.000207878477502
Fraction of train-time output set to zero:  0.30074
Fraction of test-time output set to zero:  0.0

    \end{Verbatim}

    \hypertarget{dropout-backward-pass}{%
\section{Dropout backward pass}\label{dropout-backward-pass}}

In the file \texttt{cs231n/layers.py}, implement the backward pass for
dropout. After doing so, run the following cell to numerically
gradient-check your implementation.

    \begin{tcolorbox}[breakable, size=fbox, boxrule=1pt, pad at break*=1mm,colback=cellbackground, colframe=cellborder]
\prompt{In}{incolor}{7}{\boxspacing}
\begin{Verbatim}[commandchars=\\\{\}]
\PY{n}{np}\PY{o}{.}\PY{n}{random}\PY{o}{.}\PY{n}{seed}\PY{p}{(}\PY{l+m+mi}{231}\PY{p}{)}
\PY{n}{x} \PY{o}{=} \PY{n}{np}\PY{o}{.}\PY{n}{random}\PY{o}{.}\PY{n}{randn}\PY{p}{(}\PY{l+m+mi}{10}\PY{p}{,} \PY{l+m+mi}{10}\PY{p}{)} \PY{o}{+} \PY{l+m+mi}{10}
\PY{n}{dout} \PY{o}{=} \PY{n}{np}\PY{o}{.}\PY{n}{random}\PY{o}{.}\PY{n}{randn}\PY{p}{(}\PY{o}{*}\PY{n}{x}\PY{o}{.}\PY{n}{shape}\PY{p}{)}

\PY{n}{dropout\PYZus{}param} \PY{o}{=} \PY{p}{\PYZob{}}\PY{l+s+s1}{\PYZsq{}}\PY{l+s+s1}{mode}\PY{l+s+s1}{\PYZsq{}}\PY{p}{:} \PY{l+s+s1}{\PYZsq{}}\PY{l+s+s1}{train}\PY{l+s+s1}{\PYZsq{}}\PY{p}{,} \PY{l+s+s1}{\PYZsq{}}\PY{l+s+s1}{p}\PY{l+s+s1}{\PYZsq{}}\PY{p}{:} \PY{l+m+mf}{0.2}\PY{p}{,} \PY{l+s+s1}{\PYZsq{}}\PY{l+s+s1}{seed}\PY{l+s+s1}{\PYZsq{}}\PY{p}{:} \PY{l+m+mi}{123}\PY{p}{\PYZcb{}}
\PY{n}{out}\PY{p}{,} \PY{n}{cache} \PY{o}{=} \PY{n}{dropout\PYZus{}forward}\PY{p}{(}\PY{n}{x}\PY{p}{,} \PY{n}{dropout\PYZus{}param}\PY{p}{)}
\PY{n}{dx} \PY{o}{=} \PY{n}{dropout\PYZus{}backward}\PY{p}{(}\PY{n}{dout}\PY{p}{,} \PY{n}{cache}\PY{p}{)}
\PY{n}{dx\PYZus{}num} \PY{o}{=} \PY{n}{eval\PYZus{}numerical\PYZus{}gradient\PYZus{}array}\PY{p}{(}\PY{k}{lambda} \PY{n}{xx}\PY{p}{:} \PY{n}{dropout\PYZus{}forward}\PY{p}{(}\PY{n}{xx}\PY{p}{,} \PY{n}{dropout\PYZus{}param}\PY{p}{)}\PY{p}{[}\PY{l+m+mi}{0}\PY{p}{]}\PY{p}{,} \PY{n}{x}\PY{p}{,} \PY{n}{dout}\PY{p}{)}

\PY{c+c1}{\PYZsh{} Error should be around e\PYZhy{}10 or less}
\PY{n+nb}{print}\PY{p}{(}\PY{l+s+s1}{\PYZsq{}}\PY{l+s+s1}{dx relative error: }\PY{l+s+s1}{\PYZsq{}}\PY{p}{,} \PY{n}{rel\PYZus{}error}\PY{p}{(}\PY{n}{dx}\PY{p}{,} \PY{n}{dx\PYZus{}num}\PY{p}{)}\PY{p}{)}
\end{Verbatim}
\end{tcolorbox}

    \begin{Verbatim}[commandchars=\\\{\}]
dx relative error:  5.44560814873387e-11
    \end{Verbatim}

    \hypertarget{inline-question-1}{%
\subsection{Inline Question 1:}\label{inline-question-1}}

What happens if we do not divide the values being passed through inverse
dropout by \texttt{p} in the dropout layer? Why does that happen?

\hypertarget{answer}{%
\subsection{Answer:}\label{answer}}

If we do not divide the values being passed through inverse dropout by
\(p\) in the dropout layer, the output range when training and testing
will not match. This is because if we do not divide \(p\), the expected
value of the neuron is \(px\). However, when testing, the expected value
of the neuron is \(x\). This means that the test values will be much
larger than the training values.

    \hypertarget{fully-connected-nets-with-dropout}{%
\section{Fully-connected nets with
Dropout}\label{fully-connected-nets-with-dropout}}

In the file \texttt{cs231n/classifiers/fc\_net.py}, modify your
implementation to use dropout. Specifically, if the constructor of the
network receives a value that is not 1 for the \texttt{dropout}
parameter, then the net should add a dropout layer immediately after
every ReLU nonlinearity. After doing so, run the following to
numerically gradient-check your implementation.

    \begin{tcolorbox}[breakable, size=fbox, boxrule=1pt, pad at break*=1mm,colback=cellbackground, colframe=cellborder]
\prompt{In}{incolor}{8}{\boxspacing}
\begin{Verbatim}[commandchars=\\\{\}]
\PY{n}{np}\PY{o}{.}\PY{n}{random}\PY{o}{.}\PY{n}{seed}\PY{p}{(}\PY{l+m+mi}{231}\PY{p}{)}
\PY{n}{N}\PY{p}{,} \PY{n}{D}\PY{p}{,} \PY{n}{H1}\PY{p}{,} \PY{n}{H2}\PY{p}{,} \PY{n}{C} \PY{o}{=} \PY{l+m+mi}{2}\PY{p}{,} \PY{l+m+mi}{15}\PY{p}{,} \PY{l+m+mi}{20}\PY{p}{,} \PY{l+m+mi}{30}\PY{p}{,} \PY{l+m+mi}{10}
\PY{n}{X} \PY{o}{=} \PY{n}{np}\PY{o}{.}\PY{n}{random}\PY{o}{.}\PY{n}{randn}\PY{p}{(}\PY{n}{N}\PY{p}{,} \PY{n}{D}\PY{p}{)}
\PY{n}{y} \PY{o}{=} \PY{n}{np}\PY{o}{.}\PY{n}{random}\PY{o}{.}\PY{n}{randint}\PY{p}{(}\PY{n}{C}\PY{p}{,} \PY{n}{size}\PY{o}{=}\PY{p}{(}\PY{n}{N}\PY{p}{,}\PY{p}{)}\PY{p}{)}

\PY{k}{for} \PY{n}{dropout} \PY{o+ow}{in} \PY{p}{[}\PY{l+m+mi}{1}\PY{p}{,} \PY{l+m+mf}{0.75}\PY{p}{,} \PY{l+m+mf}{0.5}\PY{p}{]}\PY{p}{:}
  \PY{n+nb}{print}\PY{p}{(}\PY{l+s+s1}{\PYZsq{}}\PY{l+s+s1}{Running check with dropout = }\PY{l+s+s1}{\PYZsq{}}\PY{p}{,} \PY{n}{dropout}\PY{p}{)}
  \PY{n}{model} \PY{o}{=} \PY{n}{FullyConnectedNet}\PY{p}{(}\PY{p}{[}\PY{n}{H1}\PY{p}{,} \PY{n}{H2}\PY{p}{]}\PY{p}{,} \PY{n}{input\PYZus{}dim}\PY{o}{=}\PY{n}{D}\PY{p}{,} \PY{n}{num\PYZus{}classes}\PY{o}{=}\PY{n}{C}\PY{p}{,}
                            \PY{n}{weight\PYZus{}scale}\PY{o}{=}\PY{l+m+mf}{5e\PYZhy{}2}\PY{p}{,} \PY{n}{dtype}\PY{o}{=}\PY{n}{np}\PY{o}{.}\PY{n}{float64}\PY{p}{,}
                            \PY{n}{dropout}\PY{o}{=}\PY{n}{dropout}\PY{p}{,} \PY{n}{seed}\PY{o}{=}\PY{l+m+mi}{123}\PY{p}{)}

  \PY{n}{loss}\PY{p}{,} \PY{n}{grads} \PY{o}{=} \PY{n}{model}\PY{o}{.}\PY{n}{loss}\PY{p}{(}\PY{n}{X}\PY{p}{,} \PY{n}{y}\PY{p}{)}
  \PY{n+nb}{print}\PY{p}{(}\PY{l+s+s1}{\PYZsq{}}\PY{l+s+s1}{Initial loss: }\PY{l+s+s1}{\PYZsq{}}\PY{p}{,} \PY{n}{loss}\PY{p}{)}
  
  \PY{c+c1}{\PYZsh{} Relative errors should be around e\PYZhy{}6 or less; Note that it\PYZsq{}s fine}
  \PY{c+c1}{\PYZsh{} if for dropout=1 you have W2 error be on the order of e\PYZhy{}5.}
  \PY{k}{for} \PY{n}{name} \PY{o+ow}{in} \PY{n+nb}{sorted}\PY{p}{(}\PY{n}{grads}\PY{p}{)}\PY{p}{:}
    \PY{n}{f} \PY{o}{=} \PY{k}{lambda} \PY{n}{\PYZus{}}\PY{p}{:} \PY{n}{model}\PY{o}{.}\PY{n}{loss}\PY{p}{(}\PY{n}{X}\PY{p}{,} \PY{n}{y}\PY{p}{)}\PY{p}{[}\PY{l+m+mi}{0}\PY{p}{]}
    \PY{n}{grad\PYZus{}num} \PY{o}{=} \PY{n}{eval\PYZus{}numerical\PYZus{}gradient}\PY{p}{(}\PY{n}{f}\PY{p}{,} \PY{n}{model}\PY{o}{.}\PY{n}{params}\PY{p}{[}\PY{n}{name}\PY{p}{]}\PY{p}{,} \PY{n}{verbose}\PY{o}{=}\PY{k+kc}{False}\PY{p}{,} \PY{n}{h}\PY{o}{=}\PY{l+m+mf}{1e\PYZhy{}5}\PY{p}{)}
    \PY{n+nb}{print}\PY{p}{(}\PY{l+s+s1}{\PYZsq{}}\PY{l+s+si}{\PYZpc{}s}\PY{l+s+s1}{ relative error: }\PY{l+s+si}{\PYZpc{}.2e}\PY{l+s+s1}{\PYZsq{}} \PY{o}{\PYZpc{}} \PY{p}{(}\PY{n}{name}\PY{p}{,} \PY{n}{rel\PYZus{}error}\PY{p}{(}\PY{n}{grad\PYZus{}num}\PY{p}{,} \PY{n}{grads}\PY{p}{[}\PY{n}{name}\PY{p}{]}\PY{p}{)}\PY{p}{)}\PY{p}{)}
  \PY{n+nb}{print}\PY{p}{(}\PY{p}{)}
\end{Verbatim}
\end{tcolorbox}

    \begin{Verbatim}[commandchars=\\\{\}]
Running check with dropout =  1
Initial loss:  2.3004790897684924
W1 relative error: 1.48e-07
W2 relative error: 2.21e-05
W3 relative error: 3.53e-07
b1 relative error: 5.38e-09
b2 relative error: 2.09e-09
b3 relative error: 5.80e-11

Running check with dropout =  0.75
Initial loss:  2.302371489704412
W1 relative error: 1.90e-07
W2 relative error: 4.76e-06
W3 relative error: 2.60e-08
b1 relative error: 4.73e-09
b2 relative error: 1.82e-09
b3 relative error: 1.70e-10

Running check with dropout =  0.5
Initial loss:  2.3042759220785896
W1 relative error: 3.11e-07
W2 relative error: 1.84e-08
W3 relative error: 5.35e-08
b1 relative error: 5.37e-09
b2 relative error: 2.99e-09
b3 relative error: 1.13e-10

    \end{Verbatim}

    \hypertarget{regularization-experiment}{%
\section{Regularization experiment}\label{regularization-experiment}}

As an experiment, we will train a pair of two-layer networks on 500
training examples: one will use no dropout, and one will use a keep
probability of 0.25. We will then visualize the training and validation
accuracies of the two networks over time.

    \begin{tcolorbox}[breakable, size=fbox, boxrule=1pt, pad at break*=1mm,colback=cellbackground, colframe=cellborder]
\prompt{In}{incolor}{9}{\boxspacing}
\begin{Verbatim}[commandchars=\\\{\}]
\PY{c+c1}{\PYZsh{} Train two identical nets, one with dropout and one without}
\PY{n}{np}\PY{o}{.}\PY{n}{random}\PY{o}{.}\PY{n}{seed}\PY{p}{(}\PY{l+m+mi}{231}\PY{p}{)}
\PY{n}{num\PYZus{}train} \PY{o}{=} \PY{l+m+mi}{500}
\PY{n}{small\PYZus{}data} \PY{o}{=} \PY{p}{\PYZob{}}
  \PY{l+s+s1}{\PYZsq{}}\PY{l+s+s1}{X\PYZus{}train}\PY{l+s+s1}{\PYZsq{}}\PY{p}{:} \PY{n}{data}\PY{p}{[}\PY{l+s+s1}{\PYZsq{}}\PY{l+s+s1}{X\PYZus{}train}\PY{l+s+s1}{\PYZsq{}}\PY{p}{]}\PY{p}{[}\PY{p}{:}\PY{n}{num\PYZus{}train}\PY{p}{]}\PY{p}{,}
  \PY{l+s+s1}{\PYZsq{}}\PY{l+s+s1}{y\PYZus{}train}\PY{l+s+s1}{\PYZsq{}}\PY{p}{:} \PY{n}{data}\PY{p}{[}\PY{l+s+s1}{\PYZsq{}}\PY{l+s+s1}{y\PYZus{}train}\PY{l+s+s1}{\PYZsq{}}\PY{p}{]}\PY{p}{[}\PY{p}{:}\PY{n}{num\PYZus{}train}\PY{p}{]}\PY{p}{,}
  \PY{l+s+s1}{\PYZsq{}}\PY{l+s+s1}{X\PYZus{}val}\PY{l+s+s1}{\PYZsq{}}\PY{p}{:} \PY{n}{data}\PY{p}{[}\PY{l+s+s1}{\PYZsq{}}\PY{l+s+s1}{X\PYZus{}val}\PY{l+s+s1}{\PYZsq{}}\PY{p}{]}\PY{p}{,}
  \PY{l+s+s1}{\PYZsq{}}\PY{l+s+s1}{y\PYZus{}val}\PY{l+s+s1}{\PYZsq{}}\PY{p}{:} \PY{n}{data}\PY{p}{[}\PY{l+s+s1}{\PYZsq{}}\PY{l+s+s1}{y\PYZus{}val}\PY{l+s+s1}{\PYZsq{}}\PY{p}{]}\PY{p}{,}
\PY{p}{\PYZcb{}}

\PY{n}{solvers} \PY{o}{=} \PY{p}{\PYZob{}}\PY{p}{\PYZcb{}}
\PY{n}{dropout\PYZus{}choices} \PY{o}{=} \PY{p}{[}\PY{l+m+mi}{1}\PY{p}{,} \PY{l+m+mf}{0.25}\PY{p}{]}
\PY{k}{for} \PY{n}{dropout} \PY{o+ow}{in} \PY{n}{dropout\PYZus{}choices}\PY{p}{:}
  \PY{n}{model} \PY{o}{=} \PY{n}{FullyConnectedNet}\PY{p}{(}\PY{p}{[}\PY{l+m+mi}{500}\PY{p}{]}\PY{p}{,} \PY{n}{dropout}\PY{o}{=}\PY{n}{dropout}\PY{p}{)}
  \PY{n+nb}{print}\PY{p}{(}\PY{n}{dropout}\PY{p}{)}

  \PY{n}{solver} \PY{o}{=} \PY{n}{Solver}\PY{p}{(}\PY{n}{model}\PY{p}{,} \PY{n}{small\PYZus{}data}\PY{p}{,}
                  \PY{n}{num\PYZus{}epochs}\PY{o}{=}\PY{l+m+mi}{25}\PY{p}{,} \PY{n}{batch\PYZus{}size}\PY{o}{=}\PY{l+m+mi}{100}\PY{p}{,}
                  \PY{n}{update\PYZus{}rule}\PY{o}{=}\PY{l+s+s1}{\PYZsq{}}\PY{l+s+s1}{adam}\PY{l+s+s1}{\PYZsq{}}\PY{p}{,}
                  \PY{n}{optim\PYZus{}config}\PY{o}{=}\PY{p}{\PYZob{}}
                    \PY{l+s+s1}{\PYZsq{}}\PY{l+s+s1}{learning\PYZus{}rate}\PY{l+s+s1}{\PYZsq{}}\PY{p}{:} \PY{l+m+mf}{5e\PYZhy{}4}\PY{p}{,}
                  \PY{p}{\PYZcb{}}\PY{p}{,}
                  \PY{n}{verbose}\PY{o}{=}\PY{k+kc}{True}\PY{p}{,} \PY{n}{print\PYZus{}every}\PY{o}{=}\PY{l+m+mi}{100}\PY{p}{)}
  \PY{n}{solver}\PY{o}{.}\PY{n}{train}\PY{p}{(}\PY{p}{)}
  \PY{n}{solvers}\PY{p}{[}\PY{n}{dropout}\PY{p}{]} \PY{o}{=} \PY{n}{solver}
  \PY{n+nb}{print}\PY{p}{(}\PY{p}{)}
\end{Verbatim}
\end{tcolorbox}

    \begin{Verbatim}[commandchars=\\\{\}]
1
(Iteration 1 / 125) loss: 7.856644
(Epoch 0 / 25) train acc: 0.260000; val\_acc: 0.184000
(Epoch 1 / 25) train acc: 0.416000; val\_acc: 0.258000
(Epoch 2 / 25) train acc: 0.482000; val\_acc: 0.276000
(Epoch 3 / 25) train acc: 0.532000; val\_acc: 0.277000
(Epoch 4 / 25) train acc: 0.600000; val\_acc: 0.271000
(Epoch 5 / 25) train acc: 0.708000; val\_acc: 0.299000
(Epoch 6 / 25) train acc: 0.722000; val\_acc: 0.282000
(Epoch 7 / 25) train acc: 0.832000; val\_acc: 0.256000
(Epoch 8 / 25) train acc: 0.878000; val\_acc: 0.268000
(Epoch 9 / 25) train acc: 0.902000; val\_acc: 0.277000
(Epoch 10 / 25) train acc: 0.896000; val\_acc: 0.262000
(Epoch 11 / 25) train acc: 0.928000; val\_acc: 0.277000
(Epoch 12 / 25) train acc: 0.962000; val\_acc: 0.297000
(Epoch 13 / 25) train acc: 0.966000; val\_acc: 0.302000
(Epoch 14 / 25) train acc: 0.972000; val\_acc: 0.317000
(Epoch 15 / 25) train acc: 0.984000; val\_acc: 0.303000
(Epoch 16 / 25) train acc: 0.994000; val\_acc: 0.301000
(Epoch 17 / 25) train acc: 0.986000; val\_acc: 0.309000
(Epoch 18 / 25) train acc: 0.990000; val\_acc: 0.305000
(Epoch 19 / 25) train acc: 0.982000; val\_acc: 0.303000
(Epoch 20 / 25) train acc: 0.978000; val\_acc: 0.301000
(Iteration 101 / 125) loss: 0.175226
(Epoch 21 / 25) train acc: 0.964000; val\_acc: 0.307000
(Epoch 22 / 25) train acc: 0.984000; val\_acc: 0.303000
(Epoch 23 / 25) train acc: 0.962000; val\_acc: 0.302000
(Epoch 24 / 25) train acc: 0.988000; val\_acc: 0.309000
(Epoch 25 / 25) train acc: 0.972000; val\_acc: 0.305000

0.25
(Iteration 1 / 125) loss: 17.318478
(Epoch 0 / 25) train acc: 0.230000; val\_acc: 0.177000
(Epoch 1 / 25) train acc: 0.378000; val\_acc: 0.243000
(Epoch 2 / 25) train acc: 0.402000; val\_acc: 0.254000
(Epoch 3 / 25) train acc: 0.502000; val\_acc: 0.276000
(Epoch 4 / 25) train acc: 0.528000; val\_acc: 0.298000
(Epoch 5 / 25) train acc: 0.562000; val\_acc: 0.297000
(Epoch 6 / 25) train acc: 0.622000; val\_acc: 0.290000
(Epoch 7 / 25) train acc: 0.626000; val\_acc: 0.298000
(Epoch 8 / 25) train acc: 0.680000; val\_acc: 0.308000
(Epoch 9 / 25) train acc: 0.720000; val\_acc: 0.295000
(Epoch 10 / 25) train acc: 0.726000; val\_acc: 0.307000
(Epoch 11 / 25) train acc: 0.746000; val\_acc: 0.307000
(Epoch 12 / 25) train acc: 0.774000; val\_acc: 0.282000
(Epoch 13 / 25) train acc: 0.808000; val\_acc: 0.301000
(Epoch 14 / 25) train acc: 0.810000; val\_acc: 0.340000
(Epoch 15 / 25) train acc: 0.850000; val\_acc: 0.350000
(Epoch 16 / 25) train acc: 0.834000; val\_acc: 0.301000
(Epoch 17 / 25) train acc: 0.838000; val\_acc: 0.290000
(Epoch 18 / 25) train acc: 0.846000; val\_acc: 0.323000
(Epoch 19 / 25) train acc: 0.860000; val\_acc: 0.323000
(Epoch 20 / 25) train acc: 0.864000; val\_acc: 0.314000
(Iteration 101 / 125) loss: 3.749460
(Epoch 21 / 25) train acc: 0.880000; val\_acc: 0.304000
(Epoch 22 / 25) train acc: 0.894000; val\_acc: 0.283000
(Epoch 23 / 25) train acc: 0.890000; val\_acc: 0.317000
(Epoch 24 / 25) train acc: 0.874000; val\_acc: 0.320000
(Epoch 25 / 25) train acc: 0.898000; val\_acc: 0.309000

    \end{Verbatim}

    \begin{tcolorbox}[breakable, size=fbox, boxrule=1pt, pad at break*=1mm,colback=cellbackground, colframe=cellborder]
\prompt{In}{incolor}{10}{\boxspacing}
\begin{Verbatim}[commandchars=\\\{\}]
\PY{c+c1}{\PYZsh{} Plot train and validation accuracies of the two models}

\PY{n}{train\PYZus{}accs} \PY{o}{=} \PY{p}{[}\PY{p}{]}
\PY{n}{val\PYZus{}accs} \PY{o}{=} \PY{p}{[}\PY{p}{]}
\PY{k}{for} \PY{n}{dropout} \PY{o+ow}{in} \PY{n}{dropout\PYZus{}choices}\PY{p}{:}
  \PY{n}{solver} \PY{o}{=} \PY{n}{solvers}\PY{p}{[}\PY{n}{dropout}\PY{p}{]}
  \PY{n}{train\PYZus{}accs}\PY{o}{.}\PY{n}{append}\PY{p}{(}\PY{n}{solver}\PY{o}{.}\PY{n}{train\PYZus{}acc\PYZus{}history}\PY{p}{[}\PY{o}{\PYZhy{}}\PY{l+m+mi}{1}\PY{p}{]}\PY{p}{)}
  \PY{n}{val\PYZus{}accs}\PY{o}{.}\PY{n}{append}\PY{p}{(}\PY{n}{solver}\PY{o}{.}\PY{n}{val\PYZus{}acc\PYZus{}history}\PY{p}{[}\PY{o}{\PYZhy{}}\PY{l+m+mi}{1}\PY{p}{]}\PY{p}{)}

\PY{n}{plt}\PY{o}{.}\PY{n}{subplot}\PY{p}{(}\PY{l+m+mi}{3}\PY{p}{,} \PY{l+m+mi}{1}\PY{p}{,} \PY{l+m+mi}{1}\PY{p}{)}
\PY{k}{for} \PY{n}{dropout} \PY{o+ow}{in} \PY{n}{dropout\PYZus{}choices}\PY{p}{:}
  \PY{n}{plt}\PY{o}{.}\PY{n}{plot}\PY{p}{(}\PY{n}{solvers}\PY{p}{[}\PY{n}{dropout}\PY{p}{]}\PY{o}{.}\PY{n}{train\PYZus{}acc\PYZus{}history}\PY{p}{,} \PY{l+s+s1}{\PYZsq{}}\PY{l+s+s1}{o}\PY{l+s+s1}{\PYZsq{}}\PY{p}{,} \PY{n}{label}\PY{o}{=}\PY{l+s+s1}{\PYZsq{}}\PY{l+s+si}{\PYZpc{}.2f}\PY{l+s+s1}{ dropout}\PY{l+s+s1}{\PYZsq{}} \PY{o}{\PYZpc{}} \PY{n}{dropout}\PY{p}{)}
\PY{n}{plt}\PY{o}{.}\PY{n}{title}\PY{p}{(}\PY{l+s+s1}{\PYZsq{}}\PY{l+s+s1}{Train accuracy}\PY{l+s+s1}{\PYZsq{}}\PY{p}{)}
\PY{n}{plt}\PY{o}{.}\PY{n}{xlabel}\PY{p}{(}\PY{l+s+s1}{\PYZsq{}}\PY{l+s+s1}{Epoch}\PY{l+s+s1}{\PYZsq{}}\PY{p}{)}
\PY{n}{plt}\PY{o}{.}\PY{n}{ylabel}\PY{p}{(}\PY{l+s+s1}{\PYZsq{}}\PY{l+s+s1}{Accuracy}\PY{l+s+s1}{\PYZsq{}}\PY{p}{)}
\PY{n}{plt}\PY{o}{.}\PY{n}{legend}\PY{p}{(}\PY{n}{ncol}\PY{o}{=}\PY{l+m+mi}{2}\PY{p}{,} \PY{n}{loc}\PY{o}{=}\PY{l+s+s1}{\PYZsq{}}\PY{l+s+s1}{lower right}\PY{l+s+s1}{\PYZsq{}}\PY{p}{)}
  
\PY{n}{plt}\PY{o}{.}\PY{n}{subplot}\PY{p}{(}\PY{l+m+mi}{3}\PY{p}{,} \PY{l+m+mi}{1}\PY{p}{,} \PY{l+m+mi}{2}\PY{p}{)}
\PY{k}{for} \PY{n}{dropout} \PY{o+ow}{in} \PY{n}{dropout\PYZus{}choices}\PY{p}{:}
  \PY{n}{plt}\PY{o}{.}\PY{n}{plot}\PY{p}{(}\PY{n}{solvers}\PY{p}{[}\PY{n}{dropout}\PY{p}{]}\PY{o}{.}\PY{n}{val\PYZus{}acc\PYZus{}history}\PY{p}{,} \PY{l+s+s1}{\PYZsq{}}\PY{l+s+s1}{o}\PY{l+s+s1}{\PYZsq{}}\PY{p}{,} \PY{n}{label}\PY{o}{=}\PY{l+s+s1}{\PYZsq{}}\PY{l+s+si}{\PYZpc{}.2f}\PY{l+s+s1}{ dropout}\PY{l+s+s1}{\PYZsq{}} \PY{o}{\PYZpc{}} \PY{n}{dropout}\PY{p}{)}
\PY{n}{plt}\PY{o}{.}\PY{n}{title}\PY{p}{(}\PY{l+s+s1}{\PYZsq{}}\PY{l+s+s1}{Val accuracy}\PY{l+s+s1}{\PYZsq{}}\PY{p}{)}
\PY{n}{plt}\PY{o}{.}\PY{n}{xlabel}\PY{p}{(}\PY{l+s+s1}{\PYZsq{}}\PY{l+s+s1}{Epoch}\PY{l+s+s1}{\PYZsq{}}\PY{p}{)}
\PY{n}{plt}\PY{o}{.}\PY{n}{ylabel}\PY{p}{(}\PY{l+s+s1}{\PYZsq{}}\PY{l+s+s1}{Accuracy}\PY{l+s+s1}{\PYZsq{}}\PY{p}{)}
\PY{n}{plt}\PY{o}{.}\PY{n}{legend}\PY{p}{(}\PY{n}{ncol}\PY{o}{=}\PY{l+m+mi}{2}\PY{p}{,} \PY{n}{loc}\PY{o}{=}\PY{l+s+s1}{\PYZsq{}}\PY{l+s+s1}{lower right}\PY{l+s+s1}{\PYZsq{}}\PY{p}{)}

\PY{n}{plt}\PY{o}{.}\PY{n}{gcf}\PY{p}{(}\PY{p}{)}\PY{o}{.}\PY{n}{set\PYZus{}size\PYZus{}inches}\PY{p}{(}\PY{l+m+mi}{15}\PY{p}{,} \PY{l+m+mi}{15}\PY{p}{)}
\PY{n}{plt}\PY{o}{.}\PY{n}{show}\PY{p}{(}\PY{p}{)}
\end{Verbatim}
\end{tcolorbox}

    \begin{center}
    \adjustimage{max size={0.9\linewidth}{0.9\paperheight}}{Dropout_files/Dropout_12_0.png}
    \end{center}
    { \hspace*{\fill} \\}
    
    \hypertarget{inline-question-2}{%
\subsection{Inline Question 2:}\label{inline-question-2}}

Compare the validation and training accuracies with and without dropout
-- what do your results suggest about dropout as a regularizer?

\hypertarget{answer}{%
\subsection{Answer:}\label{answer}}

The training accuracies without dropout is much higher than that with
dropout, but the validation accuracies indicates the inverse. The
results show that model with dropout prevents overfitting and dropout
serves as a regularizer.

    \hypertarget{inline-question-3}{%
\subsection{Inline Question 3:}\label{inline-question-3}}

Suppose we are training a deep fully-connected network for image
classification, with dropout after hidden layers (parameterized by keep
probability p). If we are concerned about overfitting, how should we
modify p (if at all) when we decide to decrease the size of the hidden
layers (that is, the number of nodes in each layer)?

\hypertarget{answer}{%
\subsection{Answer:}\label{answer}}

We should to increase \(p\) to slightly compensate the loss of neurons
such that it helps to prevent overfitting and meanwhile prevent
underfitting.

    \begin{tcolorbox}[breakable, size=fbox, boxrule=1pt, pad at break*=1mm,colback=cellbackground, colframe=cellborder]
\prompt{In}{incolor}{ }{\boxspacing}
\begin{Verbatim}[commandchars=\\\{\}]

\end{Verbatim}
\end{tcolorbox}


    % Add a bibliography block to the postdoc
    
    
    
\end{document}
