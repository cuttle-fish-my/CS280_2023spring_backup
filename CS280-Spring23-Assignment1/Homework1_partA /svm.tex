\documentclass[11pt]{article}

    \usepackage[breakable]{tcolorbox}
    \usepackage{parskip} % Stop auto-indenting (to mimic markdown behaviour)
    

    % Basic figure setup, for now with no caption control since it's done
    % automatically by Pandoc (which extracts ![](path) syntax from Markdown).
    \usepackage{graphicx}
    % Maintain compatibility with old templates. Remove in nbconvert 6.0
    \let\Oldincludegraphics\includegraphics
    % Ensure that by default, figures have no caption (until we provide a
    % proper Figure object with a Caption API and a way to capture that
    % in the conversion process - todo).
    \usepackage{caption}
    \DeclareCaptionFormat{nocaption}{}
    \captionsetup{format=nocaption,aboveskip=0pt,belowskip=0pt}

    \usepackage{float}
    \floatplacement{figure}{H} % forces figures to be placed at the correct location
    \usepackage{xcolor} % Allow colors to be defined
    \usepackage{enumerate} % Needed for markdown enumerations to work
    \usepackage{geometry} % Used to adjust the document margins
    \usepackage{amsmath} % Equations
    \usepackage{amssymb} % Equations
    \usepackage{textcomp} % defines textquotesingle
    % Hack from http://tex.stackexchange.com/a/47451/13684:
    \AtBeginDocument{%
        \def\PYZsq{\textquotesingle}% Upright quotes in Pygmentized code
    }
    \usepackage{upquote} % Upright quotes for verbatim code
    \usepackage{eurosym} % defines \euro

    \usepackage{iftex}
    \ifPDFTeX
        \usepackage[T1]{fontenc}
        \IfFileExists{alphabeta.sty}{
              \usepackage{alphabeta}
          }{
              \usepackage[mathletters]{ucs}
              \usepackage[utf8x]{inputenc}
          }
    \else
        \usepackage{fontspec}
        \usepackage{unicode-math}
    \fi

    \usepackage{fancyvrb} % verbatim replacement that allows latex
    \usepackage{grffile} % extends the file name processing of package graphics
                         % to support a larger range
    \makeatletter % fix for old versions of grffile with XeLaTeX
    \@ifpackagelater{grffile}{2019/11/01}
    {
      % Do nothing on new versions
    }
    {
      \def\Gread@@xetex#1{%
        \IfFileExists{"\Gin@base".bb}%
        {\Gread@eps{\Gin@base.bb}}%
        {\Gread@@xetex@aux#1}%
      }
    }
    \makeatother
    \usepackage[Export]{adjustbox} % Used to constrain images to a maximum size
    \adjustboxset{max size={0.9\linewidth}{0.9\paperheight}}

    % The hyperref package gives us a pdf with properly built
    % internal navigation ('pdf bookmarks' for the table of contents,
    % internal cross-reference links, web links for URLs, etc.)
    \usepackage{hyperref}
    % The default LaTeX title has an obnoxious amount of whitespace. By default,
    % titling removes some of it. It also provides customization options.
    \usepackage{titling}
    \usepackage{longtable} % longtable support required by pandoc >1.10
    \usepackage{booktabs}  % table support for pandoc > 1.12.2
    \usepackage{array}     % table support for pandoc >= 2.11.3
    \usepackage{calc}      % table minipage width calculation for pandoc >= 2.11.1
    \usepackage[inline]{enumitem} % IRkernel/repr support (it uses the enumerate* environment)
    \usepackage[normalem]{ulem} % ulem is needed to support strikethroughs (\sout)
                                % normalem makes italics be italics, not underlines
    \usepackage{mathrsfs}
    

    
    % Colors for the hyperref package
    \definecolor{urlcolor}{rgb}{0,.145,.698}
    \definecolor{linkcolor}{rgb}{.71,0.21,0.01}
    \definecolor{citecolor}{rgb}{.12,.54,.11}

    % ANSI colors
    \definecolor{ansi-black}{HTML}{3E424D}
    \definecolor{ansi-black-intense}{HTML}{282C36}
    \definecolor{ansi-red}{HTML}{E75C58}
    \definecolor{ansi-red-intense}{HTML}{B22B31}
    \definecolor{ansi-green}{HTML}{00A250}
    \definecolor{ansi-green-intense}{HTML}{007427}
    \definecolor{ansi-yellow}{HTML}{DDB62B}
    \definecolor{ansi-yellow-intense}{HTML}{B27D12}
    \definecolor{ansi-blue}{HTML}{208FFB}
    \definecolor{ansi-blue-intense}{HTML}{0065CA}
    \definecolor{ansi-magenta}{HTML}{D160C4}
    \definecolor{ansi-magenta-intense}{HTML}{A03196}
    \definecolor{ansi-cyan}{HTML}{60C6C8}
    \definecolor{ansi-cyan-intense}{HTML}{258F8F}
    \definecolor{ansi-white}{HTML}{C5C1B4}
    \definecolor{ansi-white-intense}{HTML}{A1A6B2}
    \definecolor{ansi-default-inverse-fg}{HTML}{FFFFFF}
    \definecolor{ansi-default-inverse-bg}{HTML}{000000}

    % common color for the border for error outputs.
    \definecolor{outerrorbackground}{HTML}{FFDFDF}

    % commands and environments needed by pandoc snippets
    % extracted from the output of `pandoc -s`
    \providecommand{\tightlist}{%
      \setlength{\itemsep}{0pt}\setlength{\parskip}{0pt}}
    \DefineVerbatimEnvironment{Highlighting}{Verbatim}{commandchars=\\\{\}}
    % Add ',fontsize=\small' for more characters per line
    \newenvironment{Shaded}{}{}
    \newcommand{\KeywordTok}[1]{\textcolor[rgb]{0.00,0.44,0.13}{\textbf{{#1}}}}
    \newcommand{\DataTypeTok}[1]{\textcolor[rgb]{0.56,0.13,0.00}{{#1}}}
    \newcommand{\DecValTok}[1]{\textcolor[rgb]{0.25,0.63,0.44}{{#1}}}
    \newcommand{\BaseNTok}[1]{\textcolor[rgb]{0.25,0.63,0.44}{{#1}}}
    \newcommand{\FloatTok}[1]{\textcolor[rgb]{0.25,0.63,0.44}{{#1}}}
    \newcommand{\CharTok}[1]{\textcolor[rgb]{0.25,0.44,0.63}{{#1}}}
    \newcommand{\StringTok}[1]{\textcolor[rgb]{0.25,0.44,0.63}{{#1}}}
    \newcommand{\CommentTok}[1]{\textcolor[rgb]{0.38,0.63,0.69}{\textit{{#1}}}}
    \newcommand{\OtherTok}[1]{\textcolor[rgb]{0.00,0.44,0.13}{{#1}}}
    \newcommand{\AlertTok}[1]{\textcolor[rgb]{1.00,0.00,0.00}{\textbf{{#1}}}}
    \newcommand{\FunctionTok}[1]{\textcolor[rgb]{0.02,0.16,0.49}{{#1}}}
    \newcommand{\RegionMarkerTok}[1]{{#1}}
    \newcommand{\ErrorTok}[1]{\textcolor[rgb]{1.00,0.00,0.00}{\textbf{{#1}}}}
    \newcommand{\NormalTok}[1]{{#1}}

    % Additional commands for more recent versions of Pandoc
    \newcommand{\ConstantTok}[1]{\textcolor[rgb]{0.53,0.00,0.00}{{#1}}}
    \newcommand{\SpecialCharTok}[1]{\textcolor[rgb]{0.25,0.44,0.63}{{#1}}}
    \newcommand{\VerbatimStringTok}[1]{\textcolor[rgb]{0.25,0.44,0.63}{{#1}}}
    \newcommand{\SpecialStringTok}[1]{\textcolor[rgb]{0.73,0.40,0.53}{{#1}}}
    \newcommand{\ImportTok}[1]{{#1}}
    \newcommand{\DocumentationTok}[1]{\textcolor[rgb]{0.73,0.13,0.13}{\textit{{#1}}}}
    \newcommand{\AnnotationTok}[1]{\textcolor[rgb]{0.38,0.63,0.69}{\textbf{\textit{{#1}}}}}
    \newcommand{\CommentVarTok}[1]{\textcolor[rgb]{0.38,0.63,0.69}{\textbf{\textit{{#1}}}}}
    \newcommand{\VariableTok}[1]{\textcolor[rgb]{0.10,0.09,0.49}{{#1}}}
    \newcommand{\ControlFlowTok}[1]{\textcolor[rgb]{0.00,0.44,0.13}{\textbf{{#1}}}}
    \newcommand{\OperatorTok}[1]{\textcolor[rgb]{0.40,0.40,0.40}{{#1}}}
    \newcommand{\BuiltInTok}[1]{{#1}}
    \newcommand{\ExtensionTok}[1]{{#1}}
    \newcommand{\PreprocessorTok}[1]{\textcolor[rgb]{0.74,0.48,0.00}{{#1}}}
    \newcommand{\AttributeTok}[1]{\textcolor[rgb]{0.49,0.56,0.16}{{#1}}}
    \newcommand{\InformationTok}[1]{\textcolor[rgb]{0.38,0.63,0.69}{\textbf{\textit{{#1}}}}}
    \newcommand{\WarningTok}[1]{\textcolor[rgb]{0.38,0.63,0.69}{\textbf{\textit{{#1}}}}}


    % Define a nice break command that doesn't care if a line doesn't already
    % exist.
    \def\br{\hspace*{\fill} \\* }
    % Math Jax compatibility definitions
    \def\gt{>}
    \def\lt{<}
    \let\Oldtex\TeX
    \let\Oldlatex\LaTeX
    \renewcommand{\TeX}{\textrm{\Oldtex}}
    \renewcommand{\LaTeX}{\textrm{\Oldlatex}}
    % Document parameters
    % Document title
    \title{svm}
    
    
    
    
    
% Pygments definitions
\makeatletter
\def\PY@reset{\let\PY@it=\relax \let\PY@bf=\relax%
    \let\PY@ul=\relax \let\PY@tc=\relax%
    \let\PY@bc=\relax \let\PY@ff=\relax}
\def\PY@tok#1{\csname PY@tok@#1\endcsname}
\def\PY@toks#1+{\ifx\relax#1\empty\else%
    \PY@tok{#1}\expandafter\PY@toks\fi}
\def\PY@do#1{\PY@bc{\PY@tc{\PY@ul{%
    \PY@it{\PY@bf{\PY@ff{#1}}}}}}}
\def\PY#1#2{\PY@reset\PY@toks#1+\relax+\PY@do{#2}}

\@namedef{PY@tok@w}{\def\PY@tc##1{\textcolor[rgb]{0.73,0.73,0.73}{##1}}}
\@namedef{PY@tok@c}{\let\PY@it=\textit\def\PY@tc##1{\textcolor[rgb]{0.24,0.48,0.48}{##1}}}
\@namedef{PY@tok@cp}{\def\PY@tc##1{\textcolor[rgb]{0.61,0.40,0.00}{##1}}}
\@namedef{PY@tok@k}{\let\PY@bf=\textbf\def\PY@tc##1{\textcolor[rgb]{0.00,0.50,0.00}{##1}}}
\@namedef{PY@tok@kp}{\def\PY@tc##1{\textcolor[rgb]{0.00,0.50,0.00}{##1}}}
\@namedef{PY@tok@kt}{\def\PY@tc##1{\textcolor[rgb]{0.69,0.00,0.25}{##1}}}
\@namedef{PY@tok@o}{\def\PY@tc##1{\textcolor[rgb]{0.40,0.40,0.40}{##1}}}
\@namedef{PY@tok@ow}{\let\PY@bf=\textbf\def\PY@tc##1{\textcolor[rgb]{0.67,0.13,1.00}{##1}}}
\@namedef{PY@tok@nb}{\def\PY@tc##1{\textcolor[rgb]{0.00,0.50,0.00}{##1}}}
\@namedef{PY@tok@nf}{\def\PY@tc##1{\textcolor[rgb]{0.00,0.00,1.00}{##1}}}
\@namedef{PY@tok@nc}{\let\PY@bf=\textbf\def\PY@tc##1{\textcolor[rgb]{0.00,0.00,1.00}{##1}}}
\@namedef{PY@tok@nn}{\let\PY@bf=\textbf\def\PY@tc##1{\textcolor[rgb]{0.00,0.00,1.00}{##1}}}
\@namedef{PY@tok@ne}{\let\PY@bf=\textbf\def\PY@tc##1{\textcolor[rgb]{0.80,0.25,0.22}{##1}}}
\@namedef{PY@tok@nv}{\def\PY@tc##1{\textcolor[rgb]{0.10,0.09,0.49}{##1}}}
\@namedef{PY@tok@no}{\def\PY@tc##1{\textcolor[rgb]{0.53,0.00,0.00}{##1}}}
\@namedef{PY@tok@nl}{\def\PY@tc##1{\textcolor[rgb]{0.46,0.46,0.00}{##1}}}
\@namedef{PY@tok@ni}{\let\PY@bf=\textbf\def\PY@tc##1{\textcolor[rgb]{0.44,0.44,0.44}{##1}}}
\@namedef{PY@tok@na}{\def\PY@tc##1{\textcolor[rgb]{0.41,0.47,0.13}{##1}}}
\@namedef{PY@tok@nt}{\let\PY@bf=\textbf\def\PY@tc##1{\textcolor[rgb]{0.00,0.50,0.00}{##1}}}
\@namedef{PY@tok@nd}{\def\PY@tc##1{\textcolor[rgb]{0.67,0.13,1.00}{##1}}}
\@namedef{PY@tok@s}{\def\PY@tc##1{\textcolor[rgb]{0.73,0.13,0.13}{##1}}}
\@namedef{PY@tok@sd}{\let\PY@it=\textit\def\PY@tc##1{\textcolor[rgb]{0.73,0.13,0.13}{##1}}}
\@namedef{PY@tok@si}{\let\PY@bf=\textbf\def\PY@tc##1{\textcolor[rgb]{0.64,0.35,0.47}{##1}}}
\@namedef{PY@tok@se}{\let\PY@bf=\textbf\def\PY@tc##1{\textcolor[rgb]{0.67,0.36,0.12}{##1}}}
\@namedef{PY@tok@sr}{\def\PY@tc##1{\textcolor[rgb]{0.64,0.35,0.47}{##1}}}
\@namedef{PY@tok@ss}{\def\PY@tc##1{\textcolor[rgb]{0.10,0.09,0.49}{##1}}}
\@namedef{PY@tok@sx}{\def\PY@tc##1{\textcolor[rgb]{0.00,0.50,0.00}{##1}}}
\@namedef{PY@tok@m}{\def\PY@tc##1{\textcolor[rgb]{0.40,0.40,0.40}{##1}}}
\@namedef{PY@tok@gh}{\let\PY@bf=\textbf\def\PY@tc##1{\textcolor[rgb]{0.00,0.00,0.50}{##1}}}
\@namedef{PY@tok@gu}{\let\PY@bf=\textbf\def\PY@tc##1{\textcolor[rgb]{0.50,0.00,0.50}{##1}}}
\@namedef{PY@tok@gd}{\def\PY@tc##1{\textcolor[rgb]{0.63,0.00,0.00}{##1}}}
\@namedef{PY@tok@gi}{\def\PY@tc##1{\textcolor[rgb]{0.00,0.52,0.00}{##1}}}
\@namedef{PY@tok@gr}{\def\PY@tc##1{\textcolor[rgb]{0.89,0.00,0.00}{##1}}}
\@namedef{PY@tok@ge}{\let\PY@it=\textit}
\@namedef{PY@tok@gs}{\let\PY@bf=\textbf}
\@namedef{PY@tok@gp}{\let\PY@bf=\textbf\def\PY@tc##1{\textcolor[rgb]{0.00,0.00,0.50}{##1}}}
\@namedef{PY@tok@go}{\def\PY@tc##1{\textcolor[rgb]{0.44,0.44,0.44}{##1}}}
\@namedef{PY@tok@gt}{\def\PY@tc##1{\textcolor[rgb]{0.00,0.27,0.87}{##1}}}
\@namedef{PY@tok@err}{\def\PY@bc##1{{\setlength{\fboxsep}{\string -\fboxrule}\fcolorbox[rgb]{1.00,0.00,0.00}{1,1,1}{\strut ##1}}}}
\@namedef{PY@tok@kc}{\let\PY@bf=\textbf\def\PY@tc##1{\textcolor[rgb]{0.00,0.50,0.00}{##1}}}
\@namedef{PY@tok@kd}{\let\PY@bf=\textbf\def\PY@tc##1{\textcolor[rgb]{0.00,0.50,0.00}{##1}}}
\@namedef{PY@tok@kn}{\let\PY@bf=\textbf\def\PY@tc##1{\textcolor[rgb]{0.00,0.50,0.00}{##1}}}
\@namedef{PY@tok@kr}{\let\PY@bf=\textbf\def\PY@tc##1{\textcolor[rgb]{0.00,0.50,0.00}{##1}}}
\@namedef{PY@tok@bp}{\def\PY@tc##1{\textcolor[rgb]{0.00,0.50,0.00}{##1}}}
\@namedef{PY@tok@fm}{\def\PY@tc##1{\textcolor[rgb]{0.00,0.00,1.00}{##1}}}
\@namedef{PY@tok@vc}{\def\PY@tc##1{\textcolor[rgb]{0.10,0.09,0.49}{##1}}}
\@namedef{PY@tok@vg}{\def\PY@tc##1{\textcolor[rgb]{0.10,0.09,0.49}{##1}}}
\@namedef{PY@tok@vi}{\def\PY@tc##1{\textcolor[rgb]{0.10,0.09,0.49}{##1}}}
\@namedef{PY@tok@vm}{\def\PY@tc##1{\textcolor[rgb]{0.10,0.09,0.49}{##1}}}
\@namedef{PY@tok@sa}{\def\PY@tc##1{\textcolor[rgb]{0.73,0.13,0.13}{##1}}}
\@namedef{PY@tok@sb}{\def\PY@tc##1{\textcolor[rgb]{0.73,0.13,0.13}{##1}}}
\@namedef{PY@tok@sc}{\def\PY@tc##1{\textcolor[rgb]{0.73,0.13,0.13}{##1}}}
\@namedef{PY@tok@dl}{\def\PY@tc##1{\textcolor[rgb]{0.73,0.13,0.13}{##1}}}
\@namedef{PY@tok@s2}{\def\PY@tc##1{\textcolor[rgb]{0.73,0.13,0.13}{##1}}}
\@namedef{PY@tok@sh}{\def\PY@tc##1{\textcolor[rgb]{0.73,0.13,0.13}{##1}}}
\@namedef{PY@tok@s1}{\def\PY@tc##1{\textcolor[rgb]{0.73,0.13,0.13}{##1}}}
\@namedef{PY@tok@mb}{\def\PY@tc##1{\textcolor[rgb]{0.40,0.40,0.40}{##1}}}
\@namedef{PY@tok@mf}{\def\PY@tc##1{\textcolor[rgb]{0.40,0.40,0.40}{##1}}}
\@namedef{PY@tok@mh}{\def\PY@tc##1{\textcolor[rgb]{0.40,0.40,0.40}{##1}}}
\@namedef{PY@tok@mi}{\def\PY@tc##1{\textcolor[rgb]{0.40,0.40,0.40}{##1}}}
\@namedef{PY@tok@il}{\def\PY@tc##1{\textcolor[rgb]{0.40,0.40,0.40}{##1}}}
\@namedef{PY@tok@mo}{\def\PY@tc##1{\textcolor[rgb]{0.40,0.40,0.40}{##1}}}
\@namedef{PY@tok@ch}{\let\PY@it=\textit\def\PY@tc##1{\textcolor[rgb]{0.24,0.48,0.48}{##1}}}
\@namedef{PY@tok@cm}{\let\PY@it=\textit\def\PY@tc##1{\textcolor[rgb]{0.24,0.48,0.48}{##1}}}
\@namedef{PY@tok@cpf}{\let\PY@it=\textit\def\PY@tc##1{\textcolor[rgb]{0.24,0.48,0.48}{##1}}}
\@namedef{PY@tok@c1}{\let\PY@it=\textit\def\PY@tc##1{\textcolor[rgb]{0.24,0.48,0.48}{##1}}}
\@namedef{PY@tok@cs}{\let\PY@it=\textit\def\PY@tc##1{\textcolor[rgb]{0.24,0.48,0.48}{##1}}}

\def\PYZbs{\char`\\}
\def\PYZus{\char`\_}
\def\PYZob{\char`\{}
\def\PYZcb{\char`\}}
\def\PYZca{\char`\^}
\def\PYZam{\char`\&}
\def\PYZlt{\char`\<}
\def\PYZgt{\char`\>}
\def\PYZsh{\char`\#}
\def\PYZpc{\char`\%}
\def\PYZdl{\char`\$}
\def\PYZhy{\char`\-}
\def\PYZsq{\char`\'}
\def\PYZdq{\char`\"}
\def\PYZti{\char`\~}
% for compatibility with earlier versions
\def\PYZat{@}
\def\PYZlb{[}
\def\PYZrb{]}
\makeatother


    % For linebreaks inside Verbatim environment from package fancyvrb.
    \makeatletter
        \newbox\Wrappedcontinuationbox
        \newbox\Wrappedvisiblespacebox
        \newcommand*\Wrappedvisiblespace {\textcolor{red}{\textvisiblespace}}
        \newcommand*\Wrappedcontinuationsymbol {\textcolor{red}{\llap{\tiny$\m@th\hookrightarrow$}}}
        \newcommand*\Wrappedcontinuationindent {3ex }
        \newcommand*\Wrappedafterbreak {\kern\Wrappedcontinuationindent\copy\Wrappedcontinuationbox}
        % Take advantage of the already applied Pygments mark-up to insert
        % potential linebreaks for TeX processing.
        %        {, <, #, %, $, ' and ": go to next line.
        %        _, }, ^, &, >, - and ~: stay at end of broken line.
        % Use of \textquotesingle for straight quote.
        \newcommand*\Wrappedbreaksatspecials {%
            \def\PYGZus{\discretionary{\char`\_}{\Wrappedafterbreak}{\char`\_}}%
            \def\PYGZob{\discretionary{}{\Wrappedafterbreak\char`\{}{\char`\{}}%
            \def\PYGZcb{\discretionary{\char`\}}{\Wrappedafterbreak}{\char`\}}}%
            \def\PYGZca{\discretionary{\char`\^}{\Wrappedafterbreak}{\char`\^}}%
            \def\PYGZam{\discretionary{\char`\&}{\Wrappedafterbreak}{\char`\&}}%
            \def\PYGZlt{\discretionary{}{\Wrappedafterbreak\char`\<}{\char`\<}}%
            \def\PYGZgt{\discretionary{\char`\>}{\Wrappedafterbreak}{\char`\>}}%
            \def\PYGZsh{\discretionary{}{\Wrappedafterbreak\char`\#}{\char`\#}}%
            \def\PYGZpc{\discretionary{}{\Wrappedafterbreak\char`\%}{\char`\%}}%
            \def\PYGZdl{\discretionary{}{\Wrappedafterbreak\char`\$}{\char`\$}}%
            \def\PYGZhy{\discretionary{\char`\-}{\Wrappedafterbreak}{\char`\-}}%
            \def\PYGZsq{\discretionary{}{\Wrappedafterbreak\textquotesingle}{\textquotesingle}}%
            \def\PYGZdq{\discretionary{}{\Wrappedafterbreak\char`\"}{\char`\"}}%
            \def\PYGZti{\discretionary{\char`\~}{\Wrappedafterbreak}{\char`\~}}%
        }
        % Some characters . , ; ? ! / are not pygmentized.
        % This macro makes them "active" and they will insert potential linebreaks
        \newcommand*\Wrappedbreaksatpunct {%
            \lccode`\~`\.\lowercase{\def~}{\discretionary{\hbox{\char`\.}}{\Wrappedafterbreak}{\hbox{\char`\.}}}%
            \lccode`\~`\,\lowercase{\def~}{\discretionary{\hbox{\char`\,}}{\Wrappedafterbreak}{\hbox{\char`\,}}}%
            \lccode`\~`\;\lowercase{\def~}{\discretionary{\hbox{\char`\;}}{\Wrappedafterbreak}{\hbox{\char`\;}}}%
            \lccode`\~`\:\lowercase{\def~}{\discretionary{\hbox{\char`\:}}{\Wrappedafterbreak}{\hbox{\char`\:}}}%
            \lccode`\~`\?\lowercase{\def~}{\discretionary{\hbox{\char`\?}}{\Wrappedafterbreak}{\hbox{\char`\?}}}%
            \lccode`\~`\!\lowercase{\def~}{\discretionary{\hbox{\char`\!}}{\Wrappedafterbreak}{\hbox{\char`\!}}}%
            \lccode`\~`\/\lowercase{\def~}{\discretionary{\hbox{\char`\/}}{\Wrappedafterbreak}{\hbox{\char`\/}}}%
            \catcode`\.\active
            \catcode`\,\active
            \catcode`\;\active
            \catcode`\:\active
            \catcode`\?\active
            \catcode`\!\active
            \catcode`\/\active
            \lccode`\~`\~
        }
    \makeatother

    \let\OriginalVerbatim=\Verbatim
    \makeatletter
    \renewcommand{\Verbatim}[1][1]{%
        %\parskip\z@skip
        \sbox\Wrappedcontinuationbox {\Wrappedcontinuationsymbol}%
        \sbox\Wrappedvisiblespacebox {\FV@SetupFont\Wrappedvisiblespace}%
        \def\FancyVerbFormatLine ##1{\hsize\linewidth
            \vtop{\raggedright\hyphenpenalty\z@\exhyphenpenalty\z@
                \doublehyphendemerits\z@\finalhyphendemerits\z@
                \strut ##1\strut}%
        }%
        % If the linebreak is at a space, the latter will be displayed as visible
        % space at end of first line, and a continuation symbol starts next line.
        % Stretch/shrink are however usually zero for typewriter font.
        \def\FV@Space {%
            \nobreak\hskip\z@ plus\fontdimen3\font minus\fontdimen4\font
            \discretionary{\copy\Wrappedvisiblespacebox}{\Wrappedafterbreak}
            {\kern\fontdimen2\font}%
        }%

        % Allow breaks at special characters using \PYG... macros.
        \Wrappedbreaksatspecials
        % Breaks at punctuation characters . , ; ? ! and / need catcode=\active
        \OriginalVerbatim[#1,codes*=\Wrappedbreaksatpunct]%
    }
    \makeatother

    % Exact colors from NB
    \definecolor{incolor}{HTML}{303F9F}
    \definecolor{outcolor}{HTML}{D84315}
    \definecolor{cellborder}{HTML}{CFCFCF}
    \definecolor{cellbackground}{HTML}{F7F7F7}

    % prompt
    \makeatletter
    \newcommand{\boxspacing}{\kern\kvtcb@left@rule\kern\kvtcb@boxsep}
    \makeatother
    \newcommand{\prompt}[4]{
        {\ttfamily\llap{{\color{#2}[#3]:\hspace{3pt}#4}}\vspace{-\baselineskip}}
    }
    

    
    % Prevent overflowing lines due to hard-to-break entities
    \sloppy
    % Setup hyperref package
    \hypersetup{
      breaklinks=true,  % so long urls are correctly broken across lines
      colorlinks=true,
      urlcolor=urlcolor,
      linkcolor=linkcolor,
      citecolor=citecolor,
      }
    % Slightly bigger margins than the latex defaults
    
    \geometry{verbose,tmargin=1in,bmargin=1in,lmargin=1in,rmargin=1in}
    
    

\begin{document}
    
    \maketitle
    
    

    
    \hypertarget{multiclass-support-vector-machine-exercise}{%
\section{Multiclass Support Vector Machine
exercise}\label{multiclass-support-vector-machine-exercise}}

\emph{Complete and hand in this completed worksheet (including its
outputs and any supporting code outside of the worksheet) with your
assignment submission. For more details see the
\href{http://vision.stanford.edu/teaching/cs231n/assignments.html}{assignments
page} on the course website.}

In this exercise you will:

\begin{itemize}
\tightlist
\item
  implement a fully-vectorized \textbf{loss function} for the SVM
\item
  implement the fully-vectorized expression for its \textbf{analytic
  gradient}
\item
  \textbf{check your implementation} using numerical gradient
\item
  use a validation set to \textbf{tune the learning rate and
  regularization} strength
\item
  \textbf{optimize} the loss function with \textbf{SGD}
\item
  \textbf{visualize} the final learned weights
\end{itemize}

    \begin{tcolorbox}[breakable, size=fbox, boxrule=1pt, pad at break*=1mm,colback=cellbackground, colframe=cellborder]
\prompt{In}{incolor}{1}{\boxspacing}
\begin{Verbatim}[commandchars=\\\{\}]
\PY{c+c1}{\PYZsh{} Run some setup code for this notebook.}
\PY{k+kn}{import} \PY{n+nn}{random}
\PY{k+kn}{import} \PY{n+nn}{numpy} \PY{k}{as} \PY{n+nn}{np}
\PY{k+kn}{from} \PY{n+nn}{cs231n}\PY{n+nn}{.}\PY{n+nn}{data\PYZus{}utils} \PY{k+kn}{import} \PY{n}{load\PYZus{}CIFAR10}
\PY{k+kn}{import} \PY{n+nn}{matplotlib}\PY{n+nn}{.}\PY{n+nn}{pyplot} \PY{k}{as} \PY{n+nn}{plt}

\PY{c+c1}{\PYZsh{} This is a bit of magic to make matplotlib figures appear inline in the}
\PY{c+c1}{\PYZsh{} notebook rather than in a new window.}
\PY{o}{\PYZpc{}}\PY{k}{matplotlib} inline
\PY{n}{plt}\PY{o}{.}\PY{n}{rcParams}\PY{p}{[}\PY{l+s+s1}{\PYZsq{}}\PY{l+s+s1}{figure.figsize}\PY{l+s+s1}{\PYZsq{}}\PY{p}{]} \PY{o}{=} \PY{p}{(}\PY{l+m+mf}{10.0}\PY{p}{,} \PY{l+m+mf}{8.0}\PY{p}{)} \PY{c+c1}{\PYZsh{} set default size of plots}
\PY{n}{plt}\PY{o}{.}\PY{n}{rcParams}\PY{p}{[}\PY{l+s+s1}{\PYZsq{}}\PY{l+s+s1}{image.interpolation}\PY{l+s+s1}{\PYZsq{}}\PY{p}{]} \PY{o}{=} \PY{l+s+s1}{\PYZsq{}}\PY{l+s+s1}{nearest}\PY{l+s+s1}{\PYZsq{}}
\PY{n}{plt}\PY{o}{.}\PY{n}{rcParams}\PY{p}{[}\PY{l+s+s1}{\PYZsq{}}\PY{l+s+s1}{image.cmap}\PY{l+s+s1}{\PYZsq{}}\PY{p}{]} \PY{o}{=} \PY{l+s+s1}{\PYZsq{}}\PY{l+s+s1}{gray}\PY{l+s+s1}{\PYZsq{}}

\PY{c+c1}{\PYZsh{} Some more magic so that the notebook will reload external python modules;}
\PY{c+c1}{\PYZsh{} see http://stackoverflow.com/questions/1907993/autoreload\PYZhy{}of\PYZhy{}modules\PYZhy{}in\PYZhy{}ipython}
\PY{o}{\PYZpc{}}\PY{k}{load\PYZus{}ext} autoreload
\PY{o}{\PYZpc{}}\PY{k}{autoreload} 2
\end{Verbatim}
\end{tcolorbox}

    \hypertarget{cifar-10-data-loading-and-preprocessing}{%
\subsection{CIFAR-10 Data Loading and
Preprocessing}\label{cifar-10-data-loading-and-preprocessing}}

    \begin{tcolorbox}[breakable, size=fbox, boxrule=1pt, pad at break*=1mm,colback=cellbackground, colframe=cellborder]
\prompt{In}{incolor}{2}{\boxspacing}
\begin{Verbatim}[commandchars=\\\{\}]
\PY{c+c1}{\PYZsh{} Load the raw CIFAR\PYZhy{}10 data.}
\PY{n}{cifar10\PYZus{}dir} \PY{o}{=} \PY{l+s+s1}{\PYZsq{}}\PY{l+s+s1}{cs231n/datasets/cifar\PYZhy{}10\PYZhy{}batches\PYZhy{}py}\PY{l+s+s1}{\PYZsq{}}

\PY{c+c1}{\PYZsh{} Cleaning up variables to prevent loading data multiple times (which may cause memory issue)}
\PY{k}{try}\PY{p}{:}
   \PY{k}{del} \PY{n}{X\PYZus{}train}\PY{p}{,} \PY{n}{y\PYZus{}train}
   \PY{k}{del} \PY{n}{X\PYZus{}test}\PY{p}{,} \PY{n}{y\PYZus{}test}
   \PY{n+nb}{print}\PY{p}{(}\PY{l+s+s1}{\PYZsq{}}\PY{l+s+s1}{Clear previously loaded data.}\PY{l+s+s1}{\PYZsq{}}\PY{p}{)}
\PY{k}{except}\PY{p}{:}
   \PY{k}{pass}

\PY{n}{X\PYZus{}train}\PY{p}{,} \PY{n}{y\PYZus{}train}\PY{p}{,} \PY{n}{X\PYZus{}test}\PY{p}{,} \PY{n}{y\PYZus{}test} \PY{o}{=} \PY{n}{load\PYZus{}CIFAR10}\PY{p}{(}\PY{n}{cifar10\PYZus{}dir}\PY{p}{)}

\PY{c+c1}{\PYZsh{} As a sanity check, we print out the size of the training and test data.}
\PY{n+nb}{print}\PY{p}{(}\PY{l+s+s1}{\PYZsq{}}\PY{l+s+s1}{Training data shape: }\PY{l+s+s1}{\PYZsq{}}\PY{p}{,} \PY{n}{X\PYZus{}train}\PY{o}{.}\PY{n}{shape}\PY{p}{)}
\PY{n+nb}{print}\PY{p}{(}\PY{l+s+s1}{\PYZsq{}}\PY{l+s+s1}{Training labels shape: }\PY{l+s+s1}{\PYZsq{}}\PY{p}{,} \PY{n}{y\PYZus{}train}\PY{o}{.}\PY{n}{shape}\PY{p}{)}
\PY{n+nb}{print}\PY{p}{(}\PY{l+s+s1}{\PYZsq{}}\PY{l+s+s1}{Test data shape: }\PY{l+s+s1}{\PYZsq{}}\PY{p}{,} \PY{n}{X\PYZus{}test}\PY{o}{.}\PY{n}{shape}\PY{p}{)}
\PY{n+nb}{print}\PY{p}{(}\PY{l+s+s1}{\PYZsq{}}\PY{l+s+s1}{Test labels shape: }\PY{l+s+s1}{\PYZsq{}}\PY{p}{,} \PY{n}{y\PYZus{}test}\PY{o}{.}\PY{n}{shape}\PY{p}{)}
\end{Verbatim}
\end{tcolorbox}

    \begin{Verbatim}[commandchars=\\\{\}]
Training data shape:  (50000, 32, 32, 3)
Training labels shape:  (50000,)
Test data shape:  (10000, 32, 32, 3)
Test labels shape:  (10000,)
    \end{Verbatim}

    \begin{tcolorbox}[breakable, size=fbox, boxrule=1pt, pad at break*=1mm,colback=cellbackground, colframe=cellborder]
\prompt{In}{incolor}{3}{\boxspacing}
\begin{Verbatim}[commandchars=\\\{\}]
\PY{c+c1}{\PYZsh{} Visualize some examples from the dataset.}
\PY{c+c1}{\PYZsh{} We show a few examples of training images from each class.}
\PY{n}{classes} \PY{o}{=} \PY{p}{[}\PY{l+s+s1}{\PYZsq{}}\PY{l+s+s1}{plane}\PY{l+s+s1}{\PYZsq{}}\PY{p}{,} \PY{l+s+s1}{\PYZsq{}}\PY{l+s+s1}{car}\PY{l+s+s1}{\PYZsq{}}\PY{p}{,} \PY{l+s+s1}{\PYZsq{}}\PY{l+s+s1}{bird}\PY{l+s+s1}{\PYZsq{}}\PY{p}{,} \PY{l+s+s1}{\PYZsq{}}\PY{l+s+s1}{cat}\PY{l+s+s1}{\PYZsq{}}\PY{p}{,} \PY{l+s+s1}{\PYZsq{}}\PY{l+s+s1}{deer}\PY{l+s+s1}{\PYZsq{}}\PY{p}{,} \PY{l+s+s1}{\PYZsq{}}\PY{l+s+s1}{dog}\PY{l+s+s1}{\PYZsq{}}\PY{p}{,} \PY{l+s+s1}{\PYZsq{}}\PY{l+s+s1}{frog}\PY{l+s+s1}{\PYZsq{}}\PY{p}{,} \PY{l+s+s1}{\PYZsq{}}\PY{l+s+s1}{horse}\PY{l+s+s1}{\PYZsq{}}\PY{p}{,} \PY{l+s+s1}{\PYZsq{}}\PY{l+s+s1}{ship}\PY{l+s+s1}{\PYZsq{}}\PY{p}{,} \PY{l+s+s1}{\PYZsq{}}\PY{l+s+s1}{truck}\PY{l+s+s1}{\PYZsq{}}\PY{p}{]}
\PY{n}{num\PYZus{}classes} \PY{o}{=} \PY{n+nb}{len}\PY{p}{(}\PY{n}{classes}\PY{p}{)}
\PY{n}{samples\PYZus{}per\PYZus{}class} \PY{o}{=} \PY{l+m+mi}{7}
\PY{k}{for} \PY{n}{y}\PY{p}{,} \PY{n+nb+bp}{cls} \PY{o+ow}{in} \PY{n+nb}{enumerate}\PY{p}{(}\PY{n}{classes}\PY{p}{)}\PY{p}{:}
    \PY{n}{idxs} \PY{o}{=} \PY{n}{np}\PY{o}{.}\PY{n}{flatnonzero}\PY{p}{(}\PY{n}{y\PYZus{}train} \PY{o}{==} \PY{n}{y}\PY{p}{)}
    \PY{n}{idxs} \PY{o}{=} \PY{n}{np}\PY{o}{.}\PY{n}{random}\PY{o}{.}\PY{n}{choice}\PY{p}{(}\PY{n}{idxs}\PY{p}{,} \PY{n}{samples\PYZus{}per\PYZus{}class}\PY{p}{,} \PY{n}{replace}\PY{o}{=}\PY{k+kc}{False}\PY{p}{)}
    \PY{k}{for} \PY{n}{i}\PY{p}{,} \PY{n}{idx} \PY{o+ow}{in} \PY{n+nb}{enumerate}\PY{p}{(}\PY{n}{idxs}\PY{p}{)}\PY{p}{:}
        \PY{n}{plt\PYZus{}idx} \PY{o}{=} \PY{n}{i} \PY{o}{*} \PY{n}{num\PYZus{}classes} \PY{o}{+} \PY{n}{y} \PY{o}{+} \PY{l+m+mi}{1}
        \PY{n}{plt}\PY{o}{.}\PY{n}{subplot}\PY{p}{(}\PY{n}{samples\PYZus{}per\PYZus{}class}\PY{p}{,} \PY{n}{num\PYZus{}classes}\PY{p}{,} \PY{n}{plt\PYZus{}idx}\PY{p}{)}
        \PY{n}{plt}\PY{o}{.}\PY{n}{imshow}\PY{p}{(}\PY{n}{X\PYZus{}train}\PY{p}{[}\PY{n}{idx}\PY{p}{]}\PY{o}{.}\PY{n}{astype}\PY{p}{(}\PY{l+s+s1}{\PYZsq{}}\PY{l+s+s1}{uint8}\PY{l+s+s1}{\PYZsq{}}\PY{p}{)}\PY{p}{)}
        \PY{n}{plt}\PY{o}{.}\PY{n}{axis}\PY{p}{(}\PY{l+s+s1}{\PYZsq{}}\PY{l+s+s1}{off}\PY{l+s+s1}{\PYZsq{}}\PY{p}{)}
        \PY{k}{if} \PY{n}{i} \PY{o}{==} \PY{l+m+mi}{0}\PY{p}{:}
            \PY{n}{plt}\PY{o}{.}\PY{n}{title}\PY{p}{(}\PY{n+nb+bp}{cls}\PY{p}{)}
\PY{n}{plt}\PY{o}{.}\PY{n}{show}\PY{p}{(}\PY{p}{)}
\end{Verbatim}
\end{tcolorbox}

    \begin{center}
    \adjustimage{max size={0.9\linewidth}{0.9\paperheight}}{svm_files/svm_4_0.png}
    \end{center}
    { \hspace*{\fill} \\}
    
    \begin{tcolorbox}[breakable, size=fbox, boxrule=1pt, pad at break*=1mm,colback=cellbackground, colframe=cellborder]
\prompt{In}{incolor}{4}{\boxspacing}
\begin{Verbatim}[commandchars=\\\{\}]
\PY{c+c1}{\PYZsh{} Split the data into train, val, and test sets. In addition we will}
\PY{c+c1}{\PYZsh{} create a small development set as a subset of the training data;}
\PY{c+c1}{\PYZsh{} we can use this for development so our code runs faster.}
\PY{n}{num\PYZus{}training} \PY{o}{=} \PY{l+m+mi}{49000}
\PY{n}{num\PYZus{}validation} \PY{o}{=} \PY{l+m+mi}{1000}
\PY{n}{num\PYZus{}test} \PY{o}{=} \PY{l+m+mi}{1000}
\PY{n}{num\PYZus{}dev} \PY{o}{=} \PY{l+m+mi}{500}

\PY{c+c1}{\PYZsh{} Our validation set will be num\PYZus{}validation points from the original}
\PY{c+c1}{\PYZsh{} training set.}
\PY{n}{mask} \PY{o}{=} \PY{n+nb}{range}\PY{p}{(}\PY{n}{num\PYZus{}training}\PY{p}{,} \PY{n}{num\PYZus{}training} \PY{o}{+} \PY{n}{num\PYZus{}validation}\PY{p}{)}
\PY{n}{X\PYZus{}val} \PY{o}{=} \PY{n}{X\PYZus{}train}\PY{p}{[}\PY{n}{mask}\PY{p}{]}
\PY{n}{y\PYZus{}val} \PY{o}{=} \PY{n}{y\PYZus{}train}\PY{p}{[}\PY{n}{mask}\PY{p}{]}

\PY{c+c1}{\PYZsh{} Our training set will be the first num\PYZus{}train points from the original}
\PY{c+c1}{\PYZsh{} training set.}
\PY{n}{mask} \PY{o}{=} \PY{n+nb}{range}\PY{p}{(}\PY{n}{num\PYZus{}training}\PY{p}{)}
\PY{n}{X\PYZus{}train} \PY{o}{=} \PY{n}{X\PYZus{}train}\PY{p}{[}\PY{n}{mask}\PY{p}{]}
\PY{n}{y\PYZus{}train} \PY{o}{=} \PY{n}{y\PYZus{}train}\PY{p}{[}\PY{n}{mask}\PY{p}{]}

\PY{c+c1}{\PYZsh{} We will also make a development set, which is a small subset of}
\PY{c+c1}{\PYZsh{} the training set.}
\PY{n}{mask} \PY{o}{=} \PY{n}{np}\PY{o}{.}\PY{n}{random}\PY{o}{.}\PY{n}{choice}\PY{p}{(}\PY{n}{num\PYZus{}training}\PY{p}{,} \PY{n}{num\PYZus{}dev}\PY{p}{,} \PY{n}{replace}\PY{o}{=}\PY{k+kc}{False}\PY{p}{)}
\PY{n}{X\PYZus{}dev} \PY{o}{=} \PY{n}{X\PYZus{}train}\PY{p}{[}\PY{n}{mask}\PY{p}{]}
\PY{n}{y\PYZus{}dev} \PY{o}{=} \PY{n}{y\PYZus{}train}\PY{p}{[}\PY{n}{mask}\PY{p}{]}

\PY{c+c1}{\PYZsh{} We use the first num\PYZus{}test points of the original test set as our}
\PY{c+c1}{\PYZsh{} test set.}
\PY{n}{mask} \PY{o}{=} \PY{n+nb}{range}\PY{p}{(}\PY{n}{num\PYZus{}test}\PY{p}{)}
\PY{n}{X\PYZus{}test} \PY{o}{=} \PY{n}{X\PYZus{}test}\PY{p}{[}\PY{n}{mask}\PY{p}{]}
\PY{n}{y\PYZus{}test} \PY{o}{=} \PY{n}{y\PYZus{}test}\PY{p}{[}\PY{n}{mask}\PY{p}{]}

\PY{n+nb}{print}\PY{p}{(}\PY{l+s+s1}{\PYZsq{}}\PY{l+s+s1}{Train data shape: }\PY{l+s+s1}{\PYZsq{}}\PY{p}{,} \PY{n}{X\PYZus{}train}\PY{o}{.}\PY{n}{shape}\PY{p}{)}
\PY{n+nb}{print}\PY{p}{(}\PY{l+s+s1}{\PYZsq{}}\PY{l+s+s1}{Train labels shape: }\PY{l+s+s1}{\PYZsq{}}\PY{p}{,} \PY{n}{y\PYZus{}train}\PY{o}{.}\PY{n}{shape}\PY{p}{)}
\PY{n+nb}{print}\PY{p}{(}\PY{l+s+s1}{\PYZsq{}}\PY{l+s+s1}{Validation data shape: }\PY{l+s+s1}{\PYZsq{}}\PY{p}{,} \PY{n}{X\PYZus{}val}\PY{o}{.}\PY{n}{shape}\PY{p}{)}
\PY{n+nb}{print}\PY{p}{(}\PY{l+s+s1}{\PYZsq{}}\PY{l+s+s1}{Validation labels shape: }\PY{l+s+s1}{\PYZsq{}}\PY{p}{,} \PY{n}{y\PYZus{}val}\PY{o}{.}\PY{n}{shape}\PY{p}{)}
\PY{n+nb}{print}\PY{p}{(}\PY{l+s+s1}{\PYZsq{}}\PY{l+s+s1}{Test data shape: }\PY{l+s+s1}{\PYZsq{}}\PY{p}{,} \PY{n}{X\PYZus{}test}\PY{o}{.}\PY{n}{shape}\PY{p}{)}
\PY{n+nb}{print}\PY{p}{(}\PY{l+s+s1}{\PYZsq{}}\PY{l+s+s1}{Test labels shape: }\PY{l+s+s1}{\PYZsq{}}\PY{p}{,} \PY{n}{y\PYZus{}test}\PY{o}{.}\PY{n}{shape}\PY{p}{)}
\end{Verbatim}
\end{tcolorbox}

    \begin{Verbatim}[commandchars=\\\{\}]
Train data shape:  (49000, 32, 32, 3)
Train labels shape:  (49000,)
Validation data shape:  (1000, 32, 32, 3)
Validation labels shape:  (1000,)
Test data shape:  (1000, 32, 32, 3)
Test labels shape:  (1000,)
    \end{Verbatim}

    \begin{tcolorbox}[breakable, size=fbox, boxrule=1pt, pad at break*=1mm,colback=cellbackground, colframe=cellborder]
\prompt{In}{incolor}{5}{\boxspacing}
\begin{Verbatim}[commandchars=\\\{\}]
\PY{c+c1}{\PYZsh{} Preprocessing: reshape the image data into rows}
\PY{n}{X\PYZus{}train} \PY{o}{=} \PY{n}{np}\PY{o}{.}\PY{n}{reshape}\PY{p}{(}\PY{n}{X\PYZus{}train}\PY{p}{,} \PY{p}{(}\PY{n}{X\PYZus{}train}\PY{o}{.}\PY{n}{shape}\PY{p}{[}\PY{l+m+mi}{0}\PY{p}{]}\PY{p}{,} \PY{o}{\PYZhy{}}\PY{l+m+mi}{1}\PY{p}{)}\PY{p}{)}
\PY{n}{X\PYZus{}val} \PY{o}{=} \PY{n}{np}\PY{o}{.}\PY{n}{reshape}\PY{p}{(}\PY{n}{X\PYZus{}val}\PY{p}{,} \PY{p}{(}\PY{n}{X\PYZus{}val}\PY{o}{.}\PY{n}{shape}\PY{p}{[}\PY{l+m+mi}{0}\PY{p}{]}\PY{p}{,} \PY{o}{\PYZhy{}}\PY{l+m+mi}{1}\PY{p}{)}\PY{p}{)}
\PY{n}{X\PYZus{}test} \PY{o}{=} \PY{n}{np}\PY{o}{.}\PY{n}{reshape}\PY{p}{(}\PY{n}{X\PYZus{}test}\PY{p}{,} \PY{p}{(}\PY{n}{X\PYZus{}test}\PY{o}{.}\PY{n}{shape}\PY{p}{[}\PY{l+m+mi}{0}\PY{p}{]}\PY{p}{,} \PY{o}{\PYZhy{}}\PY{l+m+mi}{1}\PY{p}{)}\PY{p}{)}
\PY{n}{X\PYZus{}dev} \PY{o}{=} \PY{n}{np}\PY{o}{.}\PY{n}{reshape}\PY{p}{(}\PY{n}{X\PYZus{}dev}\PY{p}{,} \PY{p}{(}\PY{n}{X\PYZus{}dev}\PY{o}{.}\PY{n}{shape}\PY{p}{[}\PY{l+m+mi}{0}\PY{p}{]}\PY{p}{,} \PY{o}{\PYZhy{}}\PY{l+m+mi}{1}\PY{p}{)}\PY{p}{)}

\PY{c+c1}{\PYZsh{} As a sanity check, print out the shapes of the data}
\PY{n+nb}{print}\PY{p}{(}\PY{l+s+s1}{\PYZsq{}}\PY{l+s+s1}{Training data shape: }\PY{l+s+s1}{\PYZsq{}}\PY{p}{,} \PY{n}{X\PYZus{}train}\PY{o}{.}\PY{n}{shape}\PY{p}{)}
\PY{n+nb}{print}\PY{p}{(}\PY{l+s+s1}{\PYZsq{}}\PY{l+s+s1}{Validation data shape: }\PY{l+s+s1}{\PYZsq{}}\PY{p}{,} \PY{n}{X\PYZus{}val}\PY{o}{.}\PY{n}{shape}\PY{p}{)}
\PY{n+nb}{print}\PY{p}{(}\PY{l+s+s1}{\PYZsq{}}\PY{l+s+s1}{Test data shape: }\PY{l+s+s1}{\PYZsq{}}\PY{p}{,} \PY{n}{X\PYZus{}test}\PY{o}{.}\PY{n}{shape}\PY{p}{)}
\PY{n+nb}{print}\PY{p}{(}\PY{l+s+s1}{\PYZsq{}}\PY{l+s+s1}{dev data shape: }\PY{l+s+s1}{\PYZsq{}}\PY{p}{,} \PY{n}{X\PYZus{}dev}\PY{o}{.}\PY{n}{shape}\PY{p}{)}
\end{Verbatim}
\end{tcolorbox}

    \begin{Verbatim}[commandchars=\\\{\}]
Training data shape:  (49000, 3072)
Validation data shape:  (1000, 3072)
Test data shape:  (1000, 3072)
dev data shape:  (500, 3072)
    \end{Verbatim}

    \begin{tcolorbox}[breakable, size=fbox, boxrule=1pt, pad at break*=1mm,colback=cellbackground, colframe=cellborder]
\prompt{In}{incolor}{6}{\boxspacing}
\begin{Verbatim}[commandchars=\\\{\}]
\PY{c+c1}{\PYZsh{} Preprocessing: subtract the mean image}
\PY{c+c1}{\PYZsh{} first: compute the image mean based on the training data}
\PY{n}{mean\PYZus{}image} \PY{o}{=} \PY{n}{np}\PY{o}{.}\PY{n}{mean}\PY{p}{(}\PY{n}{X\PYZus{}train}\PY{p}{,} \PY{n}{axis}\PY{o}{=}\PY{l+m+mi}{0}\PY{p}{)}
\PY{n+nb}{print}\PY{p}{(}\PY{n}{mean\PYZus{}image}\PY{p}{[}\PY{p}{:}\PY{l+m+mi}{10}\PY{p}{]}\PY{p}{)} \PY{c+c1}{\PYZsh{} print a few of the elements}
\PY{n}{plt}\PY{o}{.}\PY{n}{figure}\PY{p}{(}\PY{n}{figsize}\PY{o}{=}\PY{p}{(}\PY{l+m+mi}{4}\PY{p}{,}\PY{l+m+mi}{4}\PY{p}{)}\PY{p}{)}
\PY{n}{plt}\PY{o}{.}\PY{n}{imshow}\PY{p}{(}\PY{n}{mean\PYZus{}image}\PY{o}{.}\PY{n}{reshape}\PY{p}{(}\PY{p}{(}\PY{l+m+mi}{32}\PY{p}{,}\PY{l+m+mi}{32}\PY{p}{,}\PY{l+m+mi}{3}\PY{p}{)}\PY{p}{)}\PY{o}{.}\PY{n}{astype}\PY{p}{(}\PY{l+s+s1}{\PYZsq{}}\PY{l+s+s1}{uint8}\PY{l+s+s1}{\PYZsq{}}\PY{p}{)}\PY{p}{)} \PY{c+c1}{\PYZsh{} visualize the mean image}
\PY{n}{plt}\PY{o}{.}\PY{n}{show}\PY{p}{(}\PY{p}{)}

\PY{c+c1}{\PYZsh{} second: subtract the mean image from train and test data}
\PY{n}{X\PYZus{}train} \PY{o}{\PYZhy{}}\PY{o}{=} \PY{n}{mean\PYZus{}image}
\PY{n}{X\PYZus{}val} \PY{o}{\PYZhy{}}\PY{o}{=} \PY{n}{mean\PYZus{}image}
\PY{n}{X\PYZus{}test} \PY{o}{\PYZhy{}}\PY{o}{=} \PY{n}{mean\PYZus{}image}
\PY{n}{X\PYZus{}dev} \PY{o}{\PYZhy{}}\PY{o}{=} \PY{n}{mean\PYZus{}image}

\PY{c+c1}{\PYZsh{} third: append the bias dimension of ones (i.e. bias trick) so that our SVM}
\PY{c+c1}{\PYZsh{} only has to worry about optimizing a single weight matrix W.}
\PY{n}{X\PYZus{}train} \PY{o}{=} \PY{n}{np}\PY{o}{.}\PY{n}{hstack}\PY{p}{(}\PY{p}{[}\PY{n}{X\PYZus{}train}\PY{p}{,} \PY{n}{np}\PY{o}{.}\PY{n}{ones}\PY{p}{(}\PY{p}{(}\PY{n}{X\PYZus{}train}\PY{o}{.}\PY{n}{shape}\PY{p}{[}\PY{l+m+mi}{0}\PY{p}{]}\PY{p}{,} \PY{l+m+mi}{1}\PY{p}{)}\PY{p}{)}\PY{p}{]}\PY{p}{)}
\PY{n}{X\PYZus{}val} \PY{o}{=} \PY{n}{np}\PY{o}{.}\PY{n}{hstack}\PY{p}{(}\PY{p}{[}\PY{n}{X\PYZus{}val}\PY{p}{,} \PY{n}{np}\PY{o}{.}\PY{n}{ones}\PY{p}{(}\PY{p}{(}\PY{n}{X\PYZus{}val}\PY{o}{.}\PY{n}{shape}\PY{p}{[}\PY{l+m+mi}{0}\PY{p}{]}\PY{p}{,} \PY{l+m+mi}{1}\PY{p}{)}\PY{p}{)}\PY{p}{]}\PY{p}{)}
\PY{n}{X\PYZus{}test} \PY{o}{=} \PY{n}{np}\PY{o}{.}\PY{n}{hstack}\PY{p}{(}\PY{p}{[}\PY{n}{X\PYZus{}test}\PY{p}{,} \PY{n}{np}\PY{o}{.}\PY{n}{ones}\PY{p}{(}\PY{p}{(}\PY{n}{X\PYZus{}test}\PY{o}{.}\PY{n}{shape}\PY{p}{[}\PY{l+m+mi}{0}\PY{p}{]}\PY{p}{,} \PY{l+m+mi}{1}\PY{p}{)}\PY{p}{)}\PY{p}{]}\PY{p}{)}
\PY{n}{X\PYZus{}dev} \PY{o}{=} \PY{n}{np}\PY{o}{.}\PY{n}{hstack}\PY{p}{(}\PY{p}{[}\PY{n}{X\PYZus{}dev}\PY{p}{,} \PY{n}{np}\PY{o}{.}\PY{n}{ones}\PY{p}{(}\PY{p}{(}\PY{n}{X\PYZus{}dev}\PY{o}{.}\PY{n}{shape}\PY{p}{[}\PY{l+m+mi}{0}\PY{p}{]}\PY{p}{,} \PY{l+m+mi}{1}\PY{p}{)}\PY{p}{)}\PY{p}{]}\PY{p}{)}

\PY{n+nb}{print}\PY{p}{(}\PY{n}{X\PYZus{}train}\PY{o}{.}\PY{n}{shape}\PY{p}{,} \PY{n}{X\PYZus{}val}\PY{o}{.}\PY{n}{shape}\PY{p}{,} \PY{n}{X\PYZus{}test}\PY{o}{.}\PY{n}{shape}\PY{p}{,} \PY{n}{X\PYZus{}dev}\PY{o}{.}\PY{n}{shape}\PY{p}{)}
\end{Verbatim}
\end{tcolorbox}

    \begin{Verbatim}[commandchars=\\\{\}]
[130.64189796 135.98173469 132.47391837 130.05569388 135.34804082
 131.75402041 130.96055102 136.14328571 132.47636735 131.48467347]
    \end{Verbatim}

    \begin{center}
    \adjustimage{max size={0.9\linewidth}{0.9\paperheight}}{svm_files/svm_7_1.png}
    \end{center}
    { \hspace*{\fill} \\}
    
    \begin{Verbatim}[commandchars=\\\{\}]
(49000, 3073) (1000, 3073) (1000, 3073) (500, 3073)
    \end{Verbatim}

    \hypertarget{svm-classifier}{%
\subsection{SVM Classifier}\label{svm-classifier}}

Your code for this section will all be written inside
\texttt{cs231n/classifiers/linear\_svm.py}.

As you can see, we have prefilled the function \texttt{svm\_loss\_naive}
which uses for loops to evaluate the multiclass SVM loss function.

    \begin{tcolorbox}[breakable, size=fbox, boxrule=1pt, pad at break*=1mm,colback=cellbackground, colframe=cellborder]
\prompt{In}{incolor}{7}{\boxspacing}
\begin{Verbatim}[commandchars=\\\{\}]
\PY{c+c1}{\PYZsh{} Evaluate the naive implementation of the loss we provided for you:}
\PY{k+kn}{from} \PY{n+nn}{cs231n}\PY{n+nn}{.}\PY{n+nn}{classifiers}\PY{n+nn}{.}\PY{n+nn}{linear\PYZus{}svm} \PY{k+kn}{import} \PY{n}{svm\PYZus{}loss\PYZus{}naive}
\PY{k+kn}{import} \PY{n+nn}{time}

\PY{c+c1}{\PYZsh{} generate a random SVM weight matrix of small numbers}
\PY{n}{W} \PY{o}{=} \PY{n}{np}\PY{o}{.}\PY{n}{random}\PY{o}{.}\PY{n}{randn}\PY{p}{(}\PY{l+m+mi}{3073}\PY{p}{,} \PY{l+m+mi}{10}\PY{p}{)} \PY{o}{*} \PY{l+m+mf}{0.0001} 

\PY{n}{loss}\PY{p}{,} \PY{n}{grad} \PY{o}{=} \PY{n}{svm\PYZus{}loss\PYZus{}naive}\PY{p}{(}\PY{n}{W}\PY{p}{,} \PY{n}{X\PYZus{}dev}\PY{p}{,} \PY{n}{y\PYZus{}dev}\PY{p}{,} \PY{l+m+mf}{0.000005}\PY{p}{)}
\PY{n+nb}{print}\PY{p}{(}\PY{l+s+s1}{\PYZsq{}}\PY{l+s+s1}{loss: }\PY{l+s+si}{\PYZpc{}f}\PY{l+s+s1}{\PYZsq{}} \PY{o}{\PYZpc{}} \PY{p}{(}\PY{n}{loss}\PY{p}{,} \PY{p}{)}\PY{p}{)}
\end{Verbatim}
\end{tcolorbox}

    \begin{Verbatim}[commandchars=\\\{\}]
loss: 8.819108
    \end{Verbatim}

    The \texttt{grad} returned from the function above is right now all
zero. Derive and implement the gradient for the SVM cost function and
implement it inline inside the function \texttt{svm\_loss\_naive}. You
will find it helpful to interleave your new code inside the existing
function.

To check that you have correctly implemented the gradient correctly, you
can numerically estimate the gradient of the loss function and compare
the numeric estimate to the gradient that you computed. We have provided
code that does this for you:

    \begin{tcolorbox}[breakable, size=fbox, boxrule=1pt, pad at break*=1mm,colback=cellbackground, colframe=cellborder]
\prompt{In}{incolor}{8}{\boxspacing}
\begin{Verbatim}[commandchars=\\\{\}]
\PY{c+c1}{\PYZsh{} Once you\PYZsq{}ve implemented the gradient, recompute it with the code below}
\PY{c+c1}{\PYZsh{} and gradient check it with the function we provided for you}

\PY{c+c1}{\PYZsh{} Compute the loss and its gradient at W.}
\PY{n}{loss}\PY{p}{,} \PY{n}{grad} \PY{o}{=} \PY{n}{svm\PYZus{}loss\PYZus{}naive}\PY{p}{(}\PY{n}{W}\PY{p}{,} \PY{n}{X\PYZus{}dev}\PY{p}{,} \PY{n}{y\PYZus{}dev}\PY{p}{,} \PY{l+m+mf}{0.0}\PY{p}{)}

\PY{c+c1}{\PYZsh{} Numerically compute the gradient along several randomly chosen dimensions, and}
\PY{c+c1}{\PYZsh{} compare them with your analytically computed gradient. The numbers should match}
\PY{c+c1}{\PYZsh{} almost exactly along all dimensions.}
\PY{k+kn}{from} \PY{n+nn}{cs231n}\PY{n+nn}{.}\PY{n+nn}{gradient\PYZus{}check} \PY{k+kn}{import} \PY{n}{grad\PYZus{}check\PYZus{}sparse}
\PY{n}{f} \PY{o}{=} \PY{k}{lambda} \PY{n}{w}\PY{p}{:} \PY{n}{svm\PYZus{}loss\PYZus{}naive}\PY{p}{(}\PY{n}{w}\PY{p}{,} \PY{n}{X\PYZus{}dev}\PY{p}{,} \PY{n}{y\PYZus{}dev}\PY{p}{,} \PY{l+m+mf}{0.0}\PY{p}{)}\PY{p}{[}\PY{l+m+mi}{0}\PY{p}{]}
\PY{n}{grad\PYZus{}numerical} \PY{o}{=} \PY{n}{grad\PYZus{}check\PYZus{}sparse}\PY{p}{(}\PY{n}{f}\PY{p}{,} \PY{n}{W}\PY{p}{,} \PY{n}{grad}\PY{p}{)}

\PY{c+c1}{\PYZsh{} do the gradient check once again with regularization turned on}
\PY{c+c1}{\PYZsh{} you didn\PYZsq{}t forget the regularization gradient did you?}
\PY{n}{loss}\PY{p}{,} \PY{n}{grad} \PY{o}{=} \PY{n}{svm\PYZus{}loss\PYZus{}naive}\PY{p}{(}\PY{n}{W}\PY{p}{,} \PY{n}{X\PYZus{}dev}\PY{p}{,} \PY{n}{y\PYZus{}dev}\PY{p}{,} \PY{l+m+mf}{5e1}\PY{p}{)}
\PY{n}{f} \PY{o}{=} \PY{k}{lambda} \PY{n}{w}\PY{p}{:} \PY{n}{svm\PYZus{}loss\PYZus{}naive}\PY{p}{(}\PY{n}{w}\PY{p}{,} \PY{n}{X\PYZus{}dev}\PY{p}{,} \PY{n}{y\PYZus{}dev}\PY{p}{,} \PY{l+m+mf}{5e1}\PY{p}{)}\PY{p}{[}\PY{l+m+mi}{0}\PY{p}{]}
\PY{n}{grad\PYZus{}numerical} \PY{o}{=} \PY{n}{grad\PYZus{}check\PYZus{}sparse}\PY{p}{(}\PY{n}{f}\PY{p}{,} \PY{n}{W}\PY{p}{,} \PY{n}{grad}\PY{p}{)}
\end{Verbatim}
\end{tcolorbox}

    \begin{Verbatim}[commandchars=\\\{\}]
numerical: -3.735900 analytic: -3.735900, relative error: 5.556245e-11
numerical: 24.951147 analytic: 24.951147, relative error: 7.768410e-12
numerical: 35.782132 analytic: 35.782132, relative error: 1.492984e-12
numerical: -3.677670 analytic: -3.677670, relative error: 8.551333e-11
numerical: -7.658101 analytic: -7.658101, relative error: 3.576280e-11
numerical: -10.424697 analytic: -10.424697, relative error: 1.701058e-11
numerical: -51.779575 analytic: -51.779575, relative error: 7.049498e-12
numerical: -3.497456 analytic: -3.497456, relative error: 6.172994e-11
numerical: 2.424119 analytic: 2.424119, relative error: 2.112260e-10
numerical: -36.217647 analytic: -36.217647, relative error: 1.425886e-12
numerical: -15.355864 analytic: -15.355864, relative error: 2.438053e-11
numerical: 32.423958 analytic: 32.423958, relative error: 3.211515e-13
numerical: -45.592398 analytic: -45.592398, relative error: 1.115551e-12
numerical: 6.284818 analytic: 6.284818, relative error: 5.657302e-11
numerical: -8.972438 analytic: -8.972438, relative error: 2.407923e-11
numerical: -23.441549 analytic: -23.436703, relative error: 1.033658e-04
numerical: 19.696898 analytic: 19.696898, relative error: 1.815596e-12
numerical: 10.890563 analytic: 10.890563, relative error: 8.684859e-12
numerical: 4.603972 analytic: 4.603972, relative error: 3.167290e-11
numerical: -0.129161 analytic: -0.129161, relative error: 2.715102e-10
    \end{Verbatim}

    \textbf{Inline Question 1}

It is possible that once in a while a dimension in the gradcheck will
not match exactly. What could such a discrepancy be caused by? Is it a
reason for concern? What is a simple example in one dimension where a
gradient check could fail? How would change the margin affect of the
frequency of this happening? \emph{Hint: the SVM loss function is not
strictly speaking differentiable}

\(\color{blue}{\textit Your Answer:}\)

To begin with, the loss function is not totally differentiable at \(0\).
Since the numerical result is get from approximation, the approximation
will fail in the position that the loss function is not differentiable.

Given that the error is caused by approximation, it is not a concern.

A simple example in 1-D: consider ReLU function, \(f(x) = max(0, x)\),
if we approximate the gradient at \(x = -0.02\) and take the interval
length \(h = 0.02\), then the approximated gradient is
\[f'(x) = \frac{f(0.01)-f(-0.03)}{2\times 0.02}=\frac{1}{4} \neq 0\] but
the true gradient is \(0\).

The way to reduce the effect is to reduce the interval length \(h\).

    \begin{tcolorbox}[breakable, size=fbox, boxrule=1pt, pad at break*=1mm,colback=cellbackground, colframe=cellborder]
\prompt{In}{incolor}{9}{\boxspacing}
\begin{Verbatim}[commandchars=\\\{\}]
\PY{c+c1}{\PYZsh{} Next implement the function svm\PYZus{}loss\PYZus{}vectorized; for now only compute the loss;}
\PY{c+c1}{\PYZsh{} we will implement the gradient in a moment.}
\PY{n}{tic} \PY{o}{=} \PY{n}{time}\PY{o}{.}\PY{n}{time}\PY{p}{(}\PY{p}{)}
\PY{n}{loss\PYZus{}naive}\PY{p}{,} \PY{n}{grad\PYZus{}naive} \PY{o}{=} \PY{n}{svm\PYZus{}loss\PYZus{}naive}\PY{p}{(}\PY{n}{W}\PY{p}{,} \PY{n}{X\PYZus{}dev}\PY{p}{,} \PY{n}{y\PYZus{}dev}\PY{p}{,} \PY{l+m+mf}{0.000005}\PY{p}{)}
\PY{n}{toc} \PY{o}{=} \PY{n}{time}\PY{o}{.}\PY{n}{time}\PY{p}{(}\PY{p}{)}
\PY{n+nb}{print}\PY{p}{(}\PY{l+s+s1}{\PYZsq{}}\PY{l+s+s1}{Naive loss: }\PY{l+s+si}{\PYZpc{}e}\PY{l+s+s1}{ computed in }\PY{l+s+si}{\PYZpc{}f}\PY{l+s+s1}{s}\PY{l+s+s1}{\PYZsq{}} \PY{o}{\PYZpc{}} \PY{p}{(}\PY{n}{loss\PYZus{}naive}\PY{p}{,} \PY{n}{toc} \PY{o}{\PYZhy{}} \PY{n}{tic}\PY{p}{)}\PY{p}{)}

\PY{k+kn}{from} \PY{n+nn}{cs231n}\PY{n+nn}{.}\PY{n+nn}{classifiers}\PY{n+nn}{.}\PY{n+nn}{linear\PYZus{}svm} \PY{k+kn}{import} \PY{n}{svm\PYZus{}loss\PYZus{}vectorized}
\PY{n}{tic} \PY{o}{=} \PY{n}{time}\PY{o}{.}\PY{n}{time}\PY{p}{(}\PY{p}{)}
\PY{n}{loss\PYZus{}vectorized}\PY{p}{,} \PY{n}{\PYZus{}} \PY{o}{=} \PY{n}{svm\PYZus{}loss\PYZus{}vectorized}\PY{p}{(}\PY{n}{W}\PY{p}{,} \PY{n}{X\PYZus{}dev}\PY{p}{,} \PY{n}{y\PYZus{}dev}\PY{p}{,} \PY{l+m+mf}{0.000005}\PY{p}{)}
\PY{n}{toc} \PY{o}{=} \PY{n}{time}\PY{o}{.}\PY{n}{time}\PY{p}{(}\PY{p}{)}
\PY{n+nb}{print}\PY{p}{(}\PY{l+s+s1}{\PYZsq{}}\PY{l+s+s1}{Vectorized loss: }\PY{l+s+si}{\PYZpc{}e}\PY{l+s+s1}{ computed in }\PY{l+s+si}{\PYZpc{}f}\PY{l+s+s1}{s}\PY{l+s+s1}{\PYZsq{}} \PY{o}{\PYZpc{}} \PY{p}{(}\PY{n}{loss\PYZus{}vectorized}\PY{p}{,} \PY{n}{toc} \PY{o}{\PYZhy{}} \PY{n}{tic}\PY{p}{)}\PY{p}{)}

\PY{c+c1}{\PYZsh{} The losses should match but your vectorized implementation should be much faster.}
\PY{n+nb}{print}\PY{p}{(}\PY{l+s+s1}{\PYZsq{}}\PY{l+s+s1}{difference: }\PY{l+s+si}{\PYZpc{}f}\PY{l+s+s1}{\PYZsq{}} \PY{o}{\PYZpc{}} \PY{p}{(}\PY{n}{loss\PYZus{}naive} \PY{o}{\PYZhy{}} \PY{n}{loss\PYZus{}vectorized}\PY{p}{)}\PY{p}{)}
\end{Verbatim}
\end{tcolorbox}

    \begin{Verbatim}[commandchars=\\\{\}]
Naive loss: 8.819108e+00 computed in 0.044319s
Vectorized loss: 8.819108e+00 computed in 0.008104s
difference: 0.000000
    \end{Verbatim}

    \begin{tcolorbox}[breakable, size=fbox, boxrule=1pt, pad at break*=1mm,colback=cellbackground, colframe=cellborder]
\prompt{In}{incolor}{10}{\boxspacing}
\begin{Verbatim}[commandchars=\\\{\}]
\PY{c+c1}{\PYZsh{} Complete the implementation of svm\PYZus{}loss\PYZus{}vectorized, and compute the gradient}
\PY{c+c1}{\PYZsh{} of the loss function in a vectorized way.}

\PY{c+c1}{\PYZsh{} The naive implementation and the vectorized implementation should match, but}
\PY{c+c1}{\PYZsh{} the vectorized version should still be much faster.}
\PY{n}{tic} \PY{o}{=} \PY{n}{time}\PY{o}{.}\PY{n}{time}\PY{p}{(}\PY{p}{)}
\PY{n}{\PYZus{}}\PY{p}{,} \PY{n}{grad\PYZus{}naive} \PY{o}{=} \PY{n}{svm\PYZus{}loss\PYZus{}naive}\PY{p}{(}\PY{n}{W}\PY{p}{,} \PY{n}{X\PYZus{}dev}\PY{p}{,} \PY{n}{y\PYZus{}dev}\PY{p}{,} \PY{l+m+mf}{0.000005}\PY{p}{)}
\PY{n}{toc} \PY{o}{=} \PY{n}{time}\PY{o}{.}\PY{n}{time}\PY{p}{(}\PY{p}{)}
\PY{n+nb}{print}\PY{p}{(}\PY{l+s+s1}{\PYZsq{}}\PY{l+s+s1}{Naive loss and gradient: computed in }\PY{l+s+si}{\PYZpc{}f}\PY{l+s+s1}{s}\PY{l+s+s1}{\PYZsq{}} \PY{o}{\PYZpc{}} \PY{p}{(}\PY{n}{toc} \PY{o}{\PYZhy{}} \PY{n}{tic}\PY{p}{)}\PY{p}{)}

\PY{n}{tic} \PY{o}{=} \PY{n}{time}\PY{o}{.}\PY{n}{time}\PY{p}{(}\PY{p}{)}
\PY{n}{\PYZus{}}\PY{p}{,} \PY{n}{grad\PYZus{}vectorized} \PY{o}{=} \PY{n}{svm\PYZus{}loss\PYZus{}vectorized}\PY{p}{(}\PY{n}{W}\PY{p}{,} \PY{n}{X\PYZus{}dev}\PY{p}{,} \PY{n}{y\PYZus{}dev}\PY{p}{,} \PY{l+m+mf}{0.000005}\PY{p}{)}
\PY{n}{toc} \PY{o}{=} \PY{n}{time}\PY{o}{.}\PY{n}{time}\PY{p}{(}\PY{p}{)}
\PY{n+nb}{print}\PY{p}{(}\PY{l+s+s1}{\PYZsq{}}\PY{l+s+s1}{Vectorized loss and gradient: computed in }\PY{l+s+si}{\PYZpc{}f}\PY{l+s+s1}{s}\PY{l+s+s1}{\PYZsq{}} \PY{o}{\PYZpc{}} \PY{p}{(}\PY{n}{toc} \PY{o}{\PYZhy{}} \PY{n}{tic}\PY{p}{)}\PY{p}{)}

\PY{c+c1}{\PYZsh{} The loss is a single number, so it is easy to compare the values computed}
\PY{c+c1}{\PYZsh{} by the two implementations. The gradient on the other hand is a matrix, so}
\PY{c+c1}{\PYZsh{} we use the Frobenius norm to compare them.}
\PY{n}{difference} \PY{o}{=} \PY{n}{np}\PY{o}{.}\PY{n}{linalg}\PY{o}{.}\PY{n}{norm}\PY{p}{(}\PY{n}{grad\PYZus{}naive} \PY{o}{\PYZhy{}} \PY{n}{grad\PYZus{}vectorized}\PY{p}{,} \PY{n+nb}{ord}\PY{o}{=}\PY{l+s+s1}{\PYZsq{}}\PY{l+s+s1}{fro}\PY{l+s+s1}{\PYZsq{}}\PY{p}{)}
\PY{n+nb}{print}\PY{p}{(}\PY{l+s+s1}{\PYZsq{}}\PY{l+s+s1}{difference: }\PY{l+s+si}{\PYZpc{}f}\PY{l+s+s1}{\PYZsq{}} \PY{o}{\PYZpc{}} \PY{n}{difference}\PY{p}{)}
\end{Verbatim}
\end{tcolorbox}

    \begin{Verbatim}[commandchars=\\\{\}]
Naive loss and gradient: computed in 0.059319s
Vectorized loss and gradient: computed in 0.009089s
difference: 0.000000
    \end{Verbatim}

    \hypertarget{stochastic-gradient-descent}{%
\subsubsection{Stochastic Gradient
Descent}\label{stochastic-gradient-descent}}

We now have vectorized and efficient expressions for the loss, the
gradient and our gradient matches the numerical gradient. We are
therefore ready to do SGD to minimize the loss. Your code for this part
will be written inside
\texttt{cs231n/classifiers/linear\_classifier.py}.

    \begin{tcolorbox}[breakable, size=fbox, boxrule=1pt, pad at break*=1mm,colback=cellbackground, colframe=cellborder]
\prompt{In}{incolor}{11}{\boxspacing}
\begin{Verbatim}[commandchars=\\\{\}]
\PY{c+c1}{\PYZsh{} In the file linear\PYZus{}classifier.py, implement SGD in the function}
\PY{c+c1}{\PYZsh{} LinearClassifier.train() and then run it with the code below.}
\PY{k+kn}{from} \PY{n+nn}{cs231n}\PY{n+nn}{.}\PY{n+nn}{classifiers} \PY{k+kn}{import} \PY{n}{LinearSVM}
\PY{n}{svm} \PY{o}{=} \PY{n}{LinearSVM}\PY{p}{(}\PY{p}{)}
\PY{n}{tic} \PY{o}{=} \PY{n}{time}\PY{o}{.}\PY{n}{time}\PY{p}{(}\PY{p}{)}
\PY{n}{loss\PYZus{}hist} \PY{o}{=} \PY{n}{svm}\PY{o}{.}\PY{n}{train}\PY{p}{(}\PY{n}{X\PYZus{}train}\PY{p}{,} \PY{n}{y\PYZus{}train}\PY{p}{,} \PY{n}{learning\PYZus{}rate}\PY{o}{=}\PY{l+m+mf}{1e\PYZhy{}7}\PY{p}{,} \PY{n}{reg}\PY{o}{=}\PY{l+m+mf}{2.5e4}\PY{p}{,}
                      \PY{n}{num\PYZus{}iters}\PY{o}{=}\PY{l+m+mi}{1500}\PY{p}{,} \PY{n}{verbose}\PY{o}{=}\PY{k+kc}{True}\PY{p}{)}
\PY{n}{toc} \PY{o}{=} \PY{n}{time}\PY{o}{.}\PY{n}{time}\PY{p}{(}\PY{p}{)}
\PY{n+nb}{print}\PY{p}{(}\PY{l+s+s1}{\PYZsq{}}\PY{l+s+s1}{That took }\PY{l+s+si}{\PYZpc{}f}\PY{l+s+s1}{s}\PY{l+s+s1}{\PYZsq{}} \PY{o}{\PYZpc{}} \PY{p}{(}\PY{n}{toc} \PY{o}{\PYZhy{}} \PY{n}{tic}\PY{p}{)}\PY{p}{)}
\end{Verbatim}
\end{tcolorbox}

    \begin{Verbatim}[commandchars=\\\{\}]
iteration 0 / 1500: loss 787.759348
iteration 100 / 1500: loss 288.433866
iteration 200 / 1500: loss 108.361920
iteration 300 / 1500: loss 42.307327
iteration 400 / 1500: loss 19.240937
iteration 500 / 1500: loss 10.047169
iteration 600 / 1500: loss 7.789041
iteration 700 / 1500: loss 6.258600
iteration 800 / 1500: loss 5.857393
iteration 900 / 1500: loss 5.487107
iteration 1000 / 1500: loss 5.769576
iteration 1100 / 1500: loss 5.427533
iteration 1200 / 1500: loss 5.349334
iteration 1300 / 1500: loss 5.054104
iteration 1400 / 1500: loss 5.364794
That took 6.351572s
    \end{Verbatim}

    \begin{tcolorbox}[breakable, size=fbox, boxrule=1pt, pad at break*=1mm,colback=cellbackground, colframe=cellborder]
\prompt{In}{incolor}{12}{\boxspacing}
\begin{Verbatim}[commandchars=\\\{\}]
\PY{c+c1}{\PYZsh{} A useful debugging strategy is to plot the loss as a function of}
\PY{c+c1}{\PYZsh{} iteration number:}
\PY{n}{plt}\PY{o}{.}\PY{n}{plot}\PY{p}{(}\PY{n}{loss\PYZus{}hist}\PY{p}{)}
\PY{n}{plt}\PY{o}{.}\PY{n}{xlabel}\PY{p}{(}\PY{l+s+s1}{\PYZsq{}}\PY{l+s+s1}{Iteration number}\PY{l+s+s1}{\PYZsq{}}\PY{p}{)}
\PY{n}{plt}\PY{o}{.}\PY{n}{ylabel}\PY{p}{(}\PY{l+s+s1}{\PYZsq{}}\PY{l+s+s1}{Loss value}\PY{l+s+s1}{\PYZsq{}}\PY{p}{)}
\PY{n}{plt}\PY{o}{.}\PY{n}{show}\PY{p}{(}\PY{p}{)}
\end{Verbatim}
\end{tcolorbox}

    \begin{center}
    \adjustimage{max size={0.9\linewidth}{0.9\paperheight}}{svm_files/svm_17_0.png}
    \end{center}
    { \hspace*{\fill} \\}
    
    \begin{tcolorbox}[breakable, size=fbox, boxrule=1pt, pad at break*=1mm,colback=cellbackground, colframe=cellborder]
\prompt{In}{incolor}{13}{\boxspacing}
\begin{Verbatim}[commandchars=\\\{\}]
\PY{c+c1}{\PYZsh{} Write the LinearSVM.predict function and evaluate the performance on both the}
\PY{c+c1}{\PYZsh{} training and validation set}
\PY{n}{y\PYZus{}train\PYZus{}pred} \PY{o}{=} \PY{n}{svm}\PY{o}{.}\PY{n}{predict}\PY{p}{(}\PY{n}{X\PYZus{}train}\PY{p}{)}
\PY{n+nb}{print}\PY{p}{(}\PY{l+s+s1}{\PYZsq{}}\PY{l+s+s1}{training accuracy: }\PY{l+s+si}{\PYZpc{}f}\PY{l+s+s1}{\PYZsq{}} \PY{o}{\PYZpc{}} \PY{p}{(}\PY{n}{np}\PY{o}{.}\PY{n}{mean}\PY{p}{(}\PY{n}{y\PYZus{}train} \PY{o}{==} \PY{n}{y\PYZus{}train\PYZus{}pred}\PY{p}{)}\PY{p}{,} \PY{p}{)}\PY{p}{)}
\PY{n}{y\PYZus{}val\PYZus{}pred} \PY{o}{=} \PY{n}{svm}\PY{o}{.}\PY{n}{predict}\PY{p}{(}\PY{n}{X\PYZus{}val}\PY{p}{)}
\PY{n+nb}{print}\PY{p}{(}\PY{l+s+s1}{\PYZsq{}}\PY{l+s+s1}{validation accuracy: }\PY{l+s+si}{\PYZpc{}f}\PY{l+s+s1}{\PYZsq{}} \PY{o}{\PYZpc{}} \PY{p}{(}\PY{n}{np}\PY{o}{.}\PY{n}{mean}\PY{p}{(}\PY{n}{y\PYZus{}val} \PY{o}{==} \PY{n}{y\PYZus{}val\PYZus{}pred}\PY{p}{)}\PY{p}{,} \PY{p}{)}\PY{p}{)}
\end{Verbatim}
\end{tcolorbox}

    \begin{Verbatim}[commandchars=\\\{\}]
training accuracy: 0.363184
validation accuracy: 0.375000
    \end{Verbatim}

    \begin{tcolorbox}[breakable, size=fbox, boxrule=1pt, pad at break*=1mm,colback=cellbackground, colframe=cellborder]
\prompt{In}{incolor}{18}{\boxspacing}
\begin{Verbatim}[commandchars=\\\{\}]
\PY{c+c1}{\PYZsh{} Use the validation set to tune hyperparameters (regularization strength and}
\PY{c+c1}{\PYZsh{} learning rate). You should experiment with different ranges for the learning}
\PY{c+c1}{\PYZsh{} rates and regularization strengths; if you are careful you should be able to}
\PY{c+c1}{\PYZsh{} get a classification accuracy of about 0.39 on the validation set.}

\PY{c+c1}{\PYZsh{} Note: you may see runtime/overflow warnings during hyper\PYZhy{}parameter search. }
\PY{c+c1}{\PYZsh{} This may be caused by extreme values, and is not a bug.}

\PY{c+c1}{\PYZsh{} results is dictionary mapping tuples of the form}
\PY{c+c1}{\PYZsh{} (learning\PYZus{}rate, regularization\PYZus{}strength) to tuples of the form}
\PY{c+c1}{\PYZsh{} (training\PYZus{}accuracy, validation\PYZus{}accuracy). The accuracy is simply the fraction}
\PY{c+c1}{\PYZsh{} of data points that are correctly classified.}
\PY{n}{results} \PY{o}{=} \PY{p}{\PYZob{}}\PY{p}{\PYZcb{}}
\PY{n}{best\PYZus{}val} \PY{o}{=} \PY{o}{\PYZhy{}}\PY{l+m+mi}{1}   \PY{c+c1}{\PYZsh{} The highest validation accuracy that we have seen so far.}
\PY{n}{best\PYZus{}svm} \PY{o}{=} \PY{k+kc}{None} \PY{c+c1}{\PYZsh{} The LinearSVM object that achieved the highest validation rate.}

\PY{c+c1}{\PYZsh{}\PYZsh{}\PYZsh{}\PYZsh{}\PYZsh{}\PYZsh{}\PYZsh{}\PYZsh{}\PYZsh{}\PYZsh{}\PYZsh{}\PYZsh{}\PYZsh{}\PYZsh{}\PYZsh{}\PYZsh{}\PYZsh{}\PYZsh{}\PYZsh{}\PYZsh{}\PYZsh{}\PYZsh{}\PYZsh{}\PYZsh{}\PYZsh{}\PYZsh{}\PYZsh{}\PYZsh{}\PYZsh{}\PYZsh{}\PYZsh{}\PYZsh{}\PYZsh{}\PYZsh{}\PYZsh{}\PYZsh{}\PYZsh{}\PYZsh{}\PYZsh{}\PYZsh{}\PYZsh{}\PYZsh{}\PYZsh{}\PYZsh{}\PYZsh{}\PYZsh{}\PYZsh{}\PYZsh{}\PYZsh{}\PYZsh{}\PYZsh{}\PYZsh{}\PYZsh{}\PYZsh{}\PYZsh{}\PYZsh{}\PYZsh{}\PYZsh{}\PYZsh{}\PYZsh{}\PYZsh{}\PYZsh{}\PYZsh{}\PYZsh{}\PYZsh{}\PYZsh{}\PYZsh{}\PYZsh{}\PYZsh{}\PYZsh{}\PYZsh{}\PYZsh{}\PYZsh{}\PYZsh{}\PYZsh{}\PYZsh{}\PYZsh{}\PYZsh{}\PYZsh{}\PYZsh{}}
\PY{c+c1}{\PYZsh{} TODO:                                                                        \PYZsh{}}
\PY{c+c1}{\PYZsh{} Write code that chooses the best hyperparameters by tuning on the validation \PYZsh{}}
\PY{c+c1}{\PYZsh{} set. For each combination of hyperparameters, train a linear SVM on the      \PYZsh{}}
\PY{c+c1}{\PYZsh{} training set, compute its accuracy on the training and validation sets, and  \PYZsh{}}
\PY{c+c1}{\PYZsh{} store these numbers in the results dictionary. In addition, store the best   \PYZsh{}}
\PY{c+c1}{\PYZsh{} validation accuracy in best\PYZus{}val and the LinearSVM object that achieves this  \PYZsh{}}
\PY{c+c1}{\PYZsh{} accuracy in best\PYZus{}svm.                                                        \PYZsh{}}
\PY{c+c1}{\PYZsh{}                                                                              \PYZsh{}}
\PY{c+c1}{\PYZsh{} Hint: You should use a small value for num\PYZus{}iters as you develop your         \PYZsh{}}
\PY{c+c1}{\PYZsh{} validation code so that the SVMs don\PYZsq{}t take much time to train; once you are \PYZsh{}}
\PY{c+c1}{\PYZsh{} confident that your validation code works, you should rerun the validation   \PYZsh{}}
\PY{c+c1}{\PYZsh{} code with a larger value for num\PYZus{}iters.                                      \PYZsh{}}
\PY{c+c1}{\PYZsh{}\PYZsh{}\PYZsh{}\PYZsh{}\PYZsh{}\PYZsh{}\PYZsh{}\PYZsh{}\PYZsh{}\PYZsh{}\PYZsh{}\PYZsh{}\PYZsh{}\PYZsh{}\PYZsh{}\PYZsh{}\PYZsh{}\PYZsh{}\PYZsh{}\PYZsh{}\PYZsh{}\PYZsh{}\PYZsh{}\PYZsh{}\PYZsh{}\PYZsh{}\PYZsh{}\PYZsh{}\PYZsh{}\PYZsh{}\PYZsh{}\PYZsh{}\PYZsh{}\PYZsh{}\PYZsh{}\PYZsh{}\PYZsh{}\PYZsh{}\PYZsh{}\PYZsh{}\PYZsh{}\PYZsh{}\PYZsh{}\PYZsh{}\PYZsh{}\PYZsh{}\PYZsh{}\PYZsh{}\PYZsh{}\PYZsh{}\PYZsh{}\PYZsh{}\PYZsh{}\PYZsh{}\PYZsh{}\PYZsh{}\PYZsh{}\PYZsh{}\PYZsh{}\PYZsh{}\PYZsh{}\PYZsh{}\PYZsh{}\PYZsh{}\PYZsh{}\PYZsh{}\PYZsh{}\PYZsh{}\PYZsh{}\PYZsh{}\PYZsh{}\PYZsh{}\PYZsh{}\PYZsh{}\PYZsh{}\PYZsh{}\PYZsh{}\PYZsh{}\PYZsh{}\PYZsh{}}

\PY{c+c1}{\PYZsh{} Provided as a reference. You may or may not want to change these hyperparameters}
\PY{n}{learning\PYZus{}rates} \PY{o}{=} \PY{p}{[}\PY{l+m+mf}{1e\PYZhy{}7}\PY{p}{,} \PY{l+m+mf}{5e\PYZhy{}5}\PY{p}{]}
\PY{n}{regularization\PYZus{}strengths} \PY{o}{=} \PY{p}{[}\PY{l+m+mf}{2.5e4}\PY{p}{,} \PY{l+m+mf}{5e4}\PY{p}{]}

\PY{c+c1}{\PYZsh{} *****START OF YOUR CODE (DO NOT DELETE/MODIFY THIS LINE)*****}

\PY{k}{for} \PY{n}{lr} \PY{o+ow}{in} \PY{n}{learning\PYZus{}rates}\PY{p}{:}
    \PY{k}{for} \PY{n}{r} \PY{o+ow}{in} \PY{n}{regularization\PYZus{}strengths}\PY{p}{:}
        \PY{n}{svm} \PY{o}{=} \PY{n}{LinearSVM}\PY{p}{(}\PY{p}{)}
        \PY{n}{svm}\PY{o}{.}\PY{n}{train}\PY{p}{(}\PY{n}{X\PYZus{}train}\PY{p}{,} \PY{n}{y\PYZus{}train}\PY{p}{,} \PY{n}{learning\PYZus{}rate}\PY{o}{=}\PY{l+m+mf}{1e\PYZhy{}7}\PY{p}{,} \PY{n}{reg}\PY{o}{=}\PY{l+m+mf}{2.5e4}\PY{p}{,} \PY{n}{num\PYZus{}iters}\PY{o}{=}\PY{l+m+mi}{1500}\PY{p}{)}
        \PY{n}{y\PYZus{}train\PYZus{}pred} \PY{o}{=} \PY{n}{svm}\PY{o}{.}\PY{n}{predict}\PY{p}{(}\PY{n}{X\PYZus{}train}\PY{p}{)}
        \PY{n}{y\PYZus{}val\PYZus{}pred} \PY{o}{=} \PY{n}{svm}\PY{o}{.}\PY{n}{predict}\PY{p}{(}\PY{n}{X\PYZus{}val}\PY{p}{)}
        \PY{n}{train\PYZus{}accuracy} \PY{o}{=} \PY{n}{np}\PY{o}{.}\PY{n}{mean}\PY{p}{(}\PY{n}{y\PYZus{}train\PYZus{}pred} \PY{o}{==} \PY{n}{y\PYZus{}train}\PY{p}{)}
        \PY{n}{val\PYZus{}accuracy} \PY{o}{=} \PY{n}{np}\PY{o}{.}\PY{n}{mean}\PY{p}{(}\PY{n}{y\PYZus{}val\PYZus{}pred} \PY{o}{==} \PY{n}{y\PYZus{}val}\PY{p}{)}
        \PY{n}{results}\PY{p}{[}\PY{p}{(}\PY{n}{lr}\PY{p}{,} \PY{n}{r}\PY{p}{)}\PY{p}{]} \PY{o}{=} \PY{p}{(}\PY{n}{train\PYZus{}accuracy}\PY{p}{,} \PY{n}{val\PYZus{}accuracy}\PY{p}{)}
        \PY{k}{if} \PY{n}{val\PYZus{}accuracy} \PY{o}{\PYZgt{}} \PY{n}{best\PYZus{}val}\PY{p}{:}
            \PY{n}{best\PYZus{}val} \PY{o}{=} \PY{n}{val\PYZus{}accuracy}
            \PY{n}{best\PYZus{}svm} \PY{o}{=} \PY{n}{svm}
\PY{c+c1}{\PYZsh{} *****END OF YOUR CODE (DO NOT DELETE/MODIFY THIS LINE)*****}
    
\PY{c+c1}{\PYZsh{} Print out results.}
\PY{k}{for} \PY{n}{lr}\PY{p}{,} \PY{n}{reg} \PY{o+ow}{in} \PY{n+nb}{sorted}\PY{p}{(}\PY{n}{results}\PY{p}{)}\PY{p}{:}
    \PY{n}{train\PYZus{}accuracy}\PY{p}{,} \PY{n}{val\PYZus{}accuracy} \PY{o}{=} \PY{n}{results}\PY{p}{[}\PY{p}{(}\PY{n}{lr}\PY{p}{,} \PY{n}{reg}\PY{p}{)}\PY{p}{]}
    \PY{n+nb}{print}\PY{p}{(}\PY{l+s+s1}{\PYZsq{}}\PY{l+s+s1}{lr }\PY{l+s+si}{\PYZpc{}e}\PY{l+s+s1}{ reg }\PY{l+s+si}{\PYZpc{}e}\PY{l+s+s1}{ train accuracy: }\PY{l+s+si}{\PYZpc{}f}\PY{l+s+s1}{ val accuracy: }\PY{l+s+si}{\PYZpc{}f}\PY{l+s+s1}{\PYZsq{}} \PY{o}{\PYZpc{}} \PY{p}{(}
                \PY{n}{lr}\PY{p}{,} \PY{n}{reg}\PY{p}{,} \PY{n}{train\PYZus{}accuracy}\PY{p}{,} \PY{n}{val\PYZus{}accuracy}\PY{p}{)}\PY{p}{)}
    
\PY{n+nb}{print}\PY{p}{(}\PY{l+s+s1}{\PYZsq{}}\PY{l+s+s1}{best validation accuracy achieved during cross\PYZhy{}validation: }\PY{l+s+si}{\PYZpc{}f}\PY{l+s+s1}{\PYZsq{}} \PY{o}{\PYZpc{}} \PY{n}{best\PYZus{}val}\PY{p}{)}
\end{Verbatim}
\end{tcolorbox}

    \begin{Verbatim}[commandchars=\\\{\}]
lr 1.000000e-07 reg 2.500000e+04 train accuracy: 0.367510 val accuracy: 0.383000
lr 1.000000e-07 reg 5.000000e+04 train accuracy: 0.370939 val accuracy: 0.370000
lr 5.000000e-05 reg 2.500000e+04 train accuracy: 0.367286 val accuracy: 0.378000
lr 5.000000e-05 reg 5.000000e+04 train accuracy: 0.370327 val accuracy: 0.392000
best validation accuracy achieved during cross-validation: 0.392000
    \end{Verbatim}

    \begin{tcolorbox}[breakable, size=fbox, boxrule=1pt, pad at break*=1mm,colback=cellbackground, colframe=cellborder]
\prompt{In}{incolor}{19}{\boxspacing}
\begin{Verbatim}[commandchars=\\\{\}]
\PY{c+c1}{\PYZsh{} Visualize the cross\PYZhy{}validation results}
\PY{k+kn}{import} \PY{n+nn}{math}
\PY{k+kn}{import} \PY{n+nn}{pdb}

\PY{c+c1}{\PYZsh{} pdb.set\PYZus{}trace()}

\PY{n}{x\PYZus{}scatter} \PY{o}{=} \PY{p}{[}\PY{n}{math}\PY{o}{.}\PY{n}{log10}\PY{p}{(}\PY{n}{x}\PY{p}{[}\PY{l+m+mi}{0}\PY{p}{]}\PY{p}{)} \PY{k}{for} \PY{n}{x} \PY{o+ow}{in} \PY{n}{results}\PY{p}{]}
\PY{n}{y\PYZus{}scatter} \PY{o}{=} \PY{p}{[}\PY{n}{math}\PY{o}{.}\PY{n}{log10}\PY{p}{(}\PY{n}{x}\PY{p}{[}\PY{l+m+mi}{1}\PY{p}{]}\PY{p}{)} \PY{k}{for} \PY{n}{x} \PY{o+ow}{in} \PY{n}{results}\PY{p}{]}

\PY{c+c1}{\PYZsh{} plot training accuracy}
\PY{n}{marker\PYZus{}size} \PY{o}{=} \PY{l+m+mi}{100}
\PY{n}{colors} \PY{o}{=} \PY{p}{[}\PY{n}{results}\PY{p}{[}\PY{n}{x}\PY{p}{]}\PY{p}{[}\PY{l+m+mi}{0}\PY{p}{]} \PY{k}{for} \PY{n}{x} \PY{o+ow}{in} \PY{n}{results}\PY{p}{]}
\PY{n}{plt}\PY{o}{.}\PY{n}{subplot}\PY{p}{(}\PY{l+m+mi}{2}\PY{p}{,} \PY{l+m+mi}{1}\PY{p}{,} \PY{l+m+mi}{1}\PY{p}{)}
\PY{n}{plt}\PY{o}{.}\PY{n}{tight\PYZus{}layout}\PY{p}{(}\PY{n}{pad}\PY{o}{=}\PY{l+m+mi}{3}\PY{p}{)}
\PY{n}{plt}\PY{o}{.}\PY{n}{scatter}\PY{p}{(}\PY{n}{x\PYZus{}scatter}\PY{p}{,} \PY{n}{y\PYZus{}scatter}\PY{p}{,} \PY{n}{marker\PYZus{}size}\PY{p}{,} \PY{n}{c}\PY{o}{=}\PY{n}{colors}\PY{p}{,} \PY{n}{cmap}\PY{o}{=}\PY{n}{plt}\PY{o}{.}\PY{n}{cm}\PY{o}{.}\PY{n}{coolwarm}\PY{p}{)}
\PY{n}{plt}\PY{o}{.}\PY{n}{colorbar}\PY{p}{(}\PY{p}{)}
\PY{n}{plt}\PY{o}{.}\PY{n}{xlabel}\PY{p}{(}\PY{l+s+s1}{\PYZsq{}}\PY{l+s+s1}{log learning rate}\PY{l+s+s1}{\PYZsq{}}\PY{p}{)}
\PY{n}{plt}\PY{o}{.}\PY{n}{ylabel}\PY{p}{(}\PY{l+s+s1}{\PYZsq{}}\PY{l+s+s1}{log regularization strength}\PY{l+s+s1}{\PYZsq{}}\PY{p}{)}
\PY{n}{plt}\PY{o}{.}\PY{n}{title}\PY{p}{(}\PY{l+s+s1}{\PYZsq{}}\PY{l+s+s1}{CIFAR\PYZhy{}10 training accuracy}\PY{l+s+s1}{\PYZsq{}}\PY{p}{)}

\PY{c+c1}{\PYZsh{} plot validation accuracy}
\PY{n}{colors} \PY{o}{=} \PY{p}{[}\PY{n}{results}\PY{p}{[}\PY{n}{x}\PY{p}{]}\PY{p}{[}\PY{l+m+mi}{1}\PY{p}{]} \PY{k}{for} \PY{n}{x} \PY{o+ow}{in} \PY{n}{results}\PY{p}{]} \PY{c+c1}{\PYZsh{} default size of markers is 20}
\PY{n}{plt}\PY{o}{.}\PY{n}{subplot}\PY{p}{(}\PY{l+m+mi}{2}\PY{p}{,} \PY{l+m+mi}{1}\PY{p}{,} \PY{l+m+mi}{2}\PY{p}{)}
\PY{n}{plt}\PY{o}{.}\PY{n}{scatter}\PY{p}{(}\PY{n}{x\PYZus{}scatter}\PY{p}{,} \PY{n}{y\PYZus{}scatter}\PY{p}{,} \PY{n}{marker\PYZus{}size}\PY{p}{,} \PY{n}{c}\PY{o}{=}\PY{n}{colors}\PY{p}{,} \PY{n}{cmap}\PY{o}{=}\PY{n}{plt}\PY{o}{.}\PY{n}{cm}\PY{o}{.}\PY{n}{coolwarm}\PY{p}{)}
\PY{n}{plt}\PY{o}{.}\PY{n}{colorbar}\PY{p}{(}\PY{p}{)}
\PY{n}{plt}\PY{o}{.}\PY{n}{xlabel}\PY{p}{(}\PY{l+s+s1}{\PYZsq{}}\PY{l+s+s1}{log learning rate}\PY{l+s+s1}{\PYZsq{}}\PY{p}{)}
\PY{n}{plt}\PY{o}{.}\PY{n}{ylabel}\PY{p}{(}\PY{l+s+s1}{\PYZsq{}}\PY{l+s+s1}{log regularization strength}\PY{l+s+s1}{\PYZsq{}}\PY{p}{)}
\PY{n}{plt}\PY{o}{.}\PY{n}{title}\PY{p}{(}\PY{l+s+s1}{\PYZsq{}}\PY{l+s+s1}{CIFAR\PYZhy{}10 validation accuracy}\PY{l+s+s1}{\PYZsq{}}\PY{p}{)}
\PY{n}{plt}\PY{o}{.}\PY{n}{show}\PY{p}{(}\PY{p}{)}
\end{Verbatim}
\end{tcolorbox}

    \begin{center}
    \adjustimage{max size={0.9\linewidth}{0.9\paperheight}}{svm_files/svm_20_0.png}
    \end{center}
    { \hspace*{\fill} \\}
    
    \begin{tcolorbox}[breakable, size=fbox, boxrule=1pt, pad at break*=1mm,colback=cellbackground, colframe=cellborder]
\prompt{In}{incolor}{20}{\boxspacing}
\begin{Verbatim}[commandchars=\\\{\}]
\PY{c+c1}{\PYZsh{} Evaluate the best svm on test set}
\PY{n}{y\PYZus{}test\PYZus{}pred} \PY{o}{=} \PY{n}{best\PYZus{}svm}\PY{o}{.}\PY{n}{predict}\PY{p}{(}\PY{n}{X\PYZus{}test}\PY{p}{)}
\PY{n}{test\PYZus{}accuracy} \PY{o}{=} \PY{n}{np}\PY{o}{.}\PY{n}{mean}\PY{p}{(}\PY{n}{y\PYZus{}test} \PY{o}{==} \PY{n}{y\PYZus{}test\PYZus{}pred}\PY{p}{)}
\PY{n+nb}{print}\PY{p}{(}\PY{l+s+s1}{\PYZsq{}}\PY{l+s+s1}{linear SVM on raw pixels final test set accuracy: }\PY{l+s+si}{\PYZpc{}f}\PY{l+s+s1}{\PYZsq{}} \PY{o}{\PYZpc{}} \PY{n}{test\PYZus{}accuracy}\PY{p}{)}
\end{Verbatim}
\end{tcolorbox}

    \begin{Verbatim}[commandchars=\\\{\}]
linear SVM on raw pixels final test set accuracy: 0.381000
    \end{Verbatim}

    \begin{tcolorbox}[breakable, size=fbox, boxrule=1pt, pad at break*=1mm,colback=cellbackground, colframe=cellborder]
\prompt{In}{incolor}{21}{\boxspacing}
\begin{Verbatim}[commandchars=\\\{\}]
\PY{c+c1}{\PYZsh{} Visualize the learned weights for each class.}
\PY{c+c1}{\PYZsh{} Depending on your choice of learning rate and regularization strength, these may}
\PY{c+c1}{\PYZsh{} or may not be nice to look at.}
\PY{n}{w} \PY{o}{=} \PY{n}{best\PYZus{}svm}\PY{o}{.}\PY{n}{W}\PY{p}{[}\PY{p}{:}\PY{o}{\PYZhy{}}\PY{l+m+mi}{1}\PY{p}{,}\PY{p}{:}\PY{p}{]} \PY{c+c1}{\PYZsh{} strip out the bias}
\PY{n}{w} \PY{o}{=} \PY{n}{w}\PY{o}{.}\PY{n}{reshape}\PY{p}{(}\PY{l+m+mi}{32}\PY{p}{,} \PY{l+m+mi}{32}\PY{p}{,} \PY{l+m+mi}{3}\PY{p}{,} \PY{l+m+mi}{10}\PY{p}{)}
\PY{n}{w\PYZus{}min}\PY{p}{,} \PY{n}{w\PYZus{}max} \PY{o}{=} \PY{n}{np}\PY{o}{.}\PY{n}{min}\PY{p}{(}\PY{n}{w}\PY{p}{)}\PY{p}{,} \PY{n}{np}\PY{o}{.}\PY{n}{max}\PY{p}{(}\PY{n}{w}\PY{p}{)}
\PY{n}{classes} \PY{o}{=} \PY{p}{[}\PY{l+s+s1}{\PYZsq{}}\PY{l+s+s1}{plane}\PY{l+s+s1}{\PYZsq{}}\PY{p}{,} \PY{l+s+s1}{\PYZsq{}}\PY{l+s+s1}{car}\PY{l+s+s1}{\PYZsq{}}\PY{p}{,} \PY{l+s+s1}{\PYZsq{}}\PY{l+s+s1}{bird}\PY{l+s+s1}{\PYZsq{}}\PY{p}{,} \PY{l+s+s1}{\PYZsq{}}\PY{l+s+s1}{cat}\PY{l+s+s1}{\PYZsq{}}\PY{p}{,} \PY{l+s+s1}{\PYZsq{}}\PY{l+s+s1}{deer}\PY{l+s+s1}{\PYZsq{}}\PY{p}{,} \PY{l+s+s1}{\PYZsq{}}\PY{l+s+s1}{dog}\PY{l+s+s1}{\PYZsq{}}\PY{p}{,} \PY{l+s+s1}{\PYZsq{}}\PY{l+s+s1}{frog}\PY{l+s+s1}{\PYZsq{}}\PY{p}{,} \PY{l+s+s1}{\PYZsq{}}\PY{l+s+s1}{horse}\PY{l+s+s1}{\PYZsq{}}\PY{p}{,} \PY{l+s+s1}{\PYZsq{}}\PY{l+s+s1}{ship}\PY{l+s+s1}{\PYZsq{}}\PY{p}{,} \PY{l+s+s1}{\PYZsq{}}\PY{l+s+s1}{truck}\PY{l+s+s1}{\PYZsq{}}\PY{p}{]}
\PY{k}{for} \PY{n}{i} \PY{o+ow}{in} \PY{n+nb}{range}\PY{p}{(}\PY{l+m+mi}{10}\PY{p}{)}\PY{p}{:}
    \PY{n}{plt}\PY{o}{.}\PY{n}{subplot}\PY{p}{(}\PY{l+m+mi}{2}\PY{p}{,} \PY{l+m+mi}{5}\PY{p}{,} \PY{n}{i} \PY{o}{+} \PY{l+m+mi}{1}\PY{p}{)}
      
    \PY{c+c1}{\PYZsh{} Rescale the weights to be between 0 and 255}
    \PY{n}{wimg} \PY{o}{=} \PY{l+m+mf}{255.0} \PY{o}{*} \PY{p}{(}\PY{n}{w}\PY{p}{[}\PY{p}{:}\PY{p}{,} \PY{p}{:}\PY{p}{,} \PY{p}{:}\PY{p}{,} \PY{n}{i}\PY{p}{]}\PY{o}{.}\PY{n}{squeeze}\PY{p}{(}\PY{p}{)} \PY{o}{\PYZhy{}} \PY{n}{w\PYZus{}min}\PY{p}{)} \PY{o}{/} \PY{p}{(}\PY{n}{w\PYZus{}max} \PY{o}{\PYZhy{}} \PY{n}{w\PYZus{}min}\PY{p}{)}
    \PY{n}{plt}\PY{o}{.}\PY{n}{imshow}\PY{p}{(}\PY{n}{wimg}\PY{o}{.}\PY{n}{astype}\PY{p}{(}\PY{l+s+s1}{\PYZsq{}}\PY{l+s+s1}{uint8}\PY{l+s+s1}{\PYZsq{}}\PY{p}{)}\PY{p}{)}
    \PY{n}{plt}\PY{o}{.}\PY{n}{axis}\PY{p}{(}\PY{l+s+s1}{\PYZsq{}}\PY{l+s+s1}{off}\PY{l+s+s1}{\PYZsq{}}\PY{p}{)}
    \PY{n}{plt}\PY{o}{.}\PY{n}{title}\PY{p}{(}\PY{n}{classes}\PY{p}{[}\PY{n}{i}\PY{p}{]}\PY{p}{)}
\end{Verbatim}
\end{tcolorbox}

    \begin{center}
    \adjustimage{max size={0.9\linewidth}{0.9\paperheight}}{svm_files/svm_22_0.png}
    \end{center}
    { \hspace*{\fill} \\}
    
    \textbf{Inline question 2}

Describe what your visualized SVM weights look like, and offer a brief
explanation for why they look they way that they do.

\(\color{blue}{\textit Your Answer:}\)

Those pictures look like the ``average'' pictures of each class. This is
because the angle between \(w_k\) and \(x_i\) affects the value of
\(w_k^Tx_i\). Thus, the smaller the angle is, the more similar \(w_k\)
is to \(x_i\) and the higher the probability that \(x_i\) si classified
into class \(k\).

    \begin{tcolorbox}[breakable, size=fbox, boxrule=1pt, pad at break*=1mm,colback=cellbackground, colframe=cellborder]
\prompt{In}{incolor}{17}{\boxspacing}
\begin{Verbatim}[commandchars=\\\{\}]

\end{Verbatim}
\end{tcolorbox}


    % Add a bibliography block to the postdoc
    
    
    
\end{document}
